\section*{Résumé}

\paragraph{} Nous vivons dans un monde complexe, aliénant, en perpétuel mouvement mais qui
nous incite à l'inaction. Réseaux sociaux, smartphones, objets connectés... Aujourd'hui, la
technologie a envahi notre quotidien d'une manière à la fois ostentatoire et subtile, outil
pervers satisfaisant au moindre de nos désirs.

\paragraph{} La question peut donc être posée sans rougir : disposons-nous, à l'heure
actuelle, des moyens nécessaires pour parvenir à la création d'un système de contrôle des
masses, d'un \emph{Système Omniscient} ?

\paragraph{} Prenez le temps d'y réfléchir. Toutes les briques sont déjà présentes autour
de vous : informations personnelles, centres d'intérêt, photos de toutes sortes, relations
et interactions sociales, messagerie instantanée et \emph{micro-blogging}, géolocalisation,
cursus scolaire et professionnel, recherches en ligne et historique des consultations,
informations bancaires, médicales, pièces d'identité... Êtes-vous réellement maître de vos
données ?

\paragraph{} Notre étude sera menée en trois parties, couplant chacune un sujet de réflexion
et d'analyse avec un prototype technique qui prendra de l'ampleur au fil de l'évolution de
notre pensée.

\paragraph{} Nous étudierons dans un premier temps les différents phénomènes et modèles de
sociétés ayant conduit à une intrication si profonde de la technologie à notre quotidien. 
Nous chercherons ici à mettre l'accent sur la récolte et l'aggrégation des données
\emph{"personnelles"}.

Nous nous intéresserons ensuite aux différentes possibilités concernant la mise en place 
de manière globale d'une telle solution, notamment à travers l'utilisation d'un réseau
distribué. Nous aborderons aussi les questions et challenges architecturaux qu'une telle 
solution ne manquerait pas de soulever.

Enfin nous chargerons-nous de dotter notre système d'\emph{intelligence}. Pour cela, nous
aborderons les sujets de l'Intelligence Artificielle et du Machine Learning, expérimentant
avec différents algorithmes de traitement des données. L'éthique d'une telle solution sera
aussi abordée.

\paragraph{} Il est important de nous poser, dès aujourd'hui, les bonnes questions.\\
Qu'adviendrait-il si nos données personnelles étaient centralisées ? Si des IA interconnectées
ou des systèmes de Machine Learning distribués parvenaient à les exploiter ? Smartphones,
objets connectés, réseaux sociaux et démocratisation de la robotique : quels socles pour une
société de contrôle ?




\section*{Abstract}
Exercitation ad aliquip occaecat deserunt ipsum aliqua eiusmod incididunt. Ad excepteur mollit nostrud elit. Do qui sit sint elit ut ullamco proident incididunt eiusmod. Nisi labore ea et deserunt nulla cupidatat aute. Irure deserunt non qui eu nostrud nulla laboris nostrud irure. Amet minim officia sunt sunt do quis enim amet veniam.
