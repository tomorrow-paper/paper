\chapter{Introduction}

\paragraph{} Nous vivons dans un monde complexe, aliénant, en perpétuel mouvement mais qui
nous incite à l'inaction. Réseaux sociaux, smartphones, objets connectés... Aujourd'hui, la
technologie a envahi notre quotidien d'une manière à la fois ostentatoire et subtile, outil
pervers satisfaisant au moindre de nos désirs.

\paragraph{} La question peut donc être posée sans rougir : disposons-nous, à l'heure
actuelle, des moyens nécessaires pour parvenir à la création d'un système de contrôle des
masses, d'un \emph{Système Omniscient} ?

\paragraph{} Prenez le temps d'y réfléchir. Toutes les briques sont déjà présentes autour
de vous : informations personnelles, centres d'intérêt, photos de toutes sortes, relations
et interactions sociales, messagerie instantanée et micro-blogging, géolocalisation,
cursus scolaire et professionnel, recherches en ligne et historique des consultations,
informations bancaires, médicales, pièces d'identité... Êtes-vous réellement maître de vos
données ?

\paragraph{} Notre étude sera menée en trois parties, couplant chacune un sujet de réflexion
et d'analyse avec un prototype technique qui prendra de l'ampleur au fil de l'évolution de
notre pensée.

\paragraph{} Dans un premier temps, nous nous focaliserons sur les différents phénomènes
et modèles sociétaux ayant conduit à une intrication profonde de la technologie à notre
quotidien. Il sera question de mettre l'accent sur la récolte et l'aggrégation des données
\emph{"personnelles"} et sur les raisons qui peuvent motiver une telle collecte, pour ensuite
nous questionner sur les conséquences sur les comportements humains.

\paragraph{} Dans un second temps, notre étude portera sur les possibilités de mise en place
d'une telle solution, notamment à travers l'utilisation d'un réseau distribué. Hardware,
systèmes d'exploitation, contraintes techniques et environnementales... Quels sont les challenges
architecturaux d'une telle solution ? Nous verrons que l'Homme, à son échelle, facilite la 
construction de ces systèmes.

\paragraph{} Enfin nous chargerons-nous de doter notre prototype d'intelligence. Qu'est-ce que
l'Intelligence Artificielle ? Quelle est d'ores et déjà sa place dans nos systèmes actuels ?
Comment le Machine Learning permet-il aux innovations d'aujourd'hui de résoudre des problèmes
jusqu'alors inabordés ? Suite à l'implémentation de solutions algorithmiques complexes,
nous nous questionnerons alors sur l'éthique associée à ces sujets. Déterminisme,
eugénisme, améliorations de vie et sociales : il est important d'en questionner l'usage.

\paragraph{}  Qu'adviendrait-il si nos données personnelles étaient centralisées ? Si des IA interconnectées
ou des systèmes de Machine Learning distribués parvenaient à les exploiter ? Smartphones,
objets connectés, réseaux sociaux et démocratisation de la robotique : quels socles pour une
société de contrôle ?