\section{Les sociétés de contrôle}

\subsection*{Une portée personnelle}

\paragraph{} L'utilisation d'un service est de nos jours synonyme de production et donc collecte de données ; ne dit-on
pas que si un service est gratuit, c'est que ses utilisateurs en sont les produits ? Dès lors nous ne sommes plus maîtres
des informations que nous publions, qui peuvent être vendues ou récupérées à des fins diverses : publicité et marketing
ciblés en sont des exemples bien connus.

\paragraph{} Mais les données ne sont pas uniquement utilisées de manière ciblée. Le développement ces dernières années 
des \emph{Smart Cities} donne naissance à des initiatives auparavant impensables. Ainsi la ville de New York s'est-elle
dôtée d'un tableau de bord \cite{ProgrammableCity1} aggrégeant l'ensemble des sources de données à sa disposition pour 
mettre en exergue de nombreux indicateurs : fluctuation des prix de l'immobilier, évolution du nombre de plaintes pour 
tapage nocturne par quartier, hygiène des rues, état de la circulation aux différentes heures de la journée...
Toutes ces données dont \emph{nous} sommes la source sont ainsi mises au service de la ville.

\paragraph{} \emph{The Programmable City} \cite{ProgrammableCity0} est un projet proposant de mettre la technologie au
service de l'urbanisme. Une infrastructure et un ensemble de programmes nous permettent de piloter et d'être à l'écoute
de la ville, qui bénéficie au quotidien d'une attention nouvelle : on crée alors un cercle vertueux. Le site
\url{http://DublinDashboard.ie/} est une mise en application de ces concepts pour la ville de Dublin, proposant des
dizaines de sources de données visualisables concernant l'agglomération. D'autres initiatives, comme celle de SideWalk Labs
à Toronto \cite{ProgrammableCity3}, visent à réinventer la ville pour remettre l'humain au c\oe{}ur des processus de
décision en utilisant les nouvelles technologies et les données générées pour améliorer de manière continue les services
qu'elle fournit.

\begin{figure}[ht]
    \centering
    \includegraphics[width=300px]{chapters/01/images/programmable_city.png}
    \caption{\label{programmable_city}Le cycle de transduction des \emph{Programmable Cities}.}
\end{figure}

\paragraph{} Une donnée, ce n'est rien de plus que de l'information numérisée. Là où ces bases de données publiques
émanant de structures privées ne nous étonnent en rien - car nous sommes \emph{habitués} à les alimenter, comme l'application
Uber Movement permettant d'analyser les trajets effectués via le service Uber -, il est important de comprendre
qu'elles ne diffèrent pas de celles de nos services publics. Ce n'est qu'avec l'adoption progressive de l'\emph{Open
Data} que ces derniers se sont petit à petit ouverts au développement d'applications exploitant toutes sortes de données. 

\paragraph{} Bien évidemment, les \emph{Open Data} et autres sources de données des \emph{Smart Cities} ne sont pas
constituées de données personnelles - c'est à dire d'informations permettant d'identifier de manière directe ou non une 
personne physique \cite{PersonalData0}. Mais cela est-il réellement différent ? Ainsi la société Hitachi a-t-elle développé
en 2015 un système de prévention des crimes et délits reposant sur du machine learning \cite{ProgrammableCity2}. Une quantité
astronomique de données de toutes sortes est nécessaire pour alimenter le système, des mouvements de population aux 
antécédents de crimes enregistrés dans les zones sous surveillance. Ne peut-on pas alors craindre de voir une zone 
démographique désertée suite à un incident car son \emph{coefficient de criminalité} aura augmenté, rebutant nouveaux
résidents et jeunes parents à s'y installer ?

\paragraph{} La finalité de ces systèmes peut être réduite à cette simple question : se savoir surveillé restreint-il
les instincts pervers, ou au contraire cela les désinhibe-t-il ?

\paragraph{} \emph{Veiller} nous vient du latin classique \emph{vigilare} \guillemotleft être éveillé, être attentif, sur
ses gardes\guillemotright, mais aussi \guillemotleft entourer de soins\guillemotright \cite{Surveillance0}. Le préfixé
\emph{Surveiller} (XVI\up{ème} siècle) signifie alors \guillemotleft contrôler, observer avec attention\guillemotright.
Être \emph{sous la surveillance de}, c'est être \emph{observé par} mais aussi \emph{contrôlé par}, c'est-à-dire subordonné
directement ou indirectement à quelque chose ou quelqu'un.

\paragraph{} L'\emph{instinct} vient du latin \emph{instinctus}, qui dispose d'un double sens. En latin classique, il 
signifie \guillemotleft instigation, excitation, impulsion\guillemotright, et en latin chrétien \guillemotleft penchant,
tendance naturelle\guillemotright \cite{Instinct0}. Théorisé par les philosophes, le mot désigne pour Montaigne (XVI\up{ème} siècle) 
\guillemotleft l'ensemble des pulsions naturelles qui régissent le comportement animal ou humain\guillemotright (cette 
notion de \emph{pulsion} se retrouve chez Freud avec le \emph{Trieb} allemand), tandis que Pascal (XVII\up{ème} siècle)
en fait \guillemotleft la faculté naturelle de sentir, de deviner\guillemotright.

\paragraph{} Enfin, \emph{pervertir} apparaît dans l'ancien français au XII\up{ème} siècle sous la forme \emph{purvertir} et est
dérivé du latin \emph{pervertire} ; \emph{per-} \guillemotleft l'action de\guillemotright, et \emph{vertere} \guillemotleft
tourner, verser\guillemotright : il signifie alors \guillemotleft mettre sens dessus-dessous, faire mal tourner\guillemotright
\cite{Pervers0}. Dans le langage religieux, le \emph{pervers} est une personne \guillemotleft portée à faire le mal\guillemotright
et est alors synonyme de \emph{dur}, \emph{cruel}, \emph{furieux} (XIII\up{ème} siècle). En français moderne, il désigne 
\guillemotleft la personne qui montre une tendance pathologique à accomplir des actes immoraux\guillemotright, voire même
celle \guillemotleft dont les \emph{pulsions} ne sont pas fixées\guillemotright pour les psychanalystes.

\paragraph{} Qu'entend-on donc par \emph{instincts pervers} ? L'instinct, c'est la pulsion ; la perversion, c'est l'absence
de fixation. Or \emph{fixer}, c'est \guillemotleft établir d'une manière durable dans une position déterminée\guillemotright \cite{Fixe0}.
Les instincts pervers, ce sont donc toutes les pulsions immorales, "hors norme", hors du cadre déterminé, en décalage avec la réalité
telle qu'elle est acceptée - ou définie - par le groupe ; la perversion n'a de sens que dans un contexte sociétal.

\paragraph{} Nous touchons donc bien à la raison d'être des sociétés de contrôle : restreindre les instincts pervers c'est
normaliser, niveller par le bas, et l'ensemble des systèmes de surveillance contribuent à accentuer ce phénomène \cite{SocialMedia0}.
\emph{Se savoir surveillé}, ce n'est rien de plus que posséder un compte sur un réseau social : on y incarne à tour de
\emph{rôle} juge, avocat et accusé, et l'on s'accroche à l'idée illusoire que l'on n'a pas perdu le \emph{contrôle}.

\paragraph{} Vaudrait-il mieux alors vivre dans une société où sévissent quelques individus extrêmement pervers, ou dans
une société \emph{lissée}, \emph{amortie}, \emph{normalisée} ? Le modèle panoptique (voir \ref{panoptique}, page \pageref{panoptique})
ne s'est jamais généralisé car il était trop évident, manquant de subtilité. Les technologies ont permis de mettre en application
un modèle \emph{surveillant-surveillé}.


\subsection*{Évolution du partage}

\paragraph{} Partager, c'est \guillemotleft diviser en parts destinées à être distribuées\guillemotright \cite{Partage0}.
Un héritage ou un trésor ; un repas ou un moment - le partage est un processus fondamentalement \emph{inclusif} qui nous
\emph{lie} à l'autre.

\paragraph{} Cependant les anglo-saxons ont fait de l'\emph{information} un terme invariablement singulier : il n'en existe qu'une.
Or c'est bien cette dernière qui est de nos jours la plus partagée, et donc la plus \emph{divisée}. En ligne, le partage 
ne nous implique pas en tant qu'individu prenant part à une action, mais en tant que relai dans la diffusion de l'information. 
Cette évolution des m\oe{}urs a participé à redéfinir nos interactions sociales en mettant le partage d'information au 
c\oe{}ur de nos échanges.

\paragraph{} Diluée dans la masse, elle perd son sens. Si l'on s'en réfère à la figure \ref{social_media_sharing} ci-dessous,
on constate qu'une part importante des informations publiées sur les réseaux sociaux - en dehors des interactions directes 
avec les membres du réseau - ont pour sujet le Moi. À l'origine une approche inclusive, le partage s'est mué en démarche 
égocentrique compulsive.

\begin{figure}[ht]
    \centering
    \includegraphics[width=400px]{chapters/01/images/social_media_sharing.png}
    \caption{\label{social_media_sharing}Principaux sujets de publication des adultes américains durant l'année 2016. \cite{SocialMedia1}.}
\end{figure}

\paragraph{} L'une des conséquences directe de cette mutation est l'appauvrissement des échanges tenus sur ces plateformes :
on n'argumente plus, mais l'on émet des \emph{opinions}. Les réactions sont bien souvent faites à chaud, ce qui entraine
invariablement une radicalisation des positions. Ce climat crée alors un terreau fertile pour la prolifération d'informations
fausses, détournées ou sorties de leur contexte d'origine dans le seul but de faire \emph{réagir}.

\paragraph{} Et l'objectif est alors atteint : l'individu \emph{s'}est nivellé par le bas, normalisé. Car la réaction,
comme nous l'avons vu en introduction de ce chapitre, c'est l'abandon de sa puissance : quand il réagit, l'individu
ne s'inscrit plus dans une démarche constructive. Il se subordonne au déclencheur de sa réaction, qui seul en tire profit.

\paragraph{} Même s'il n'est pas directement ressenti comme tel, le partage est aujourd'hui devenu un rapport de force. 
Le publicateur d'une information est \emph{surveillé} par ceux qui, peut-être, y réagiront. Mais il est aussi lui-même
\emph{surveillant} des publications des autres, prompt à y réagir. 


\subsection*{L'individu technique}

\paragraph{} En définitive, l'apport de la technique nous a permis de mettre en application ce que Foucault envisageait
comme finalité du modèle Panoptique... sur nous-même. Nous sommes désormais ammenés à conceptualiser l'individu en termes
technologiques, comme un \emph{individu technique}, un \emph{profil}. Les informations nous concernant rentrent bien
gentiment dans des cases et nos pensées - jusqu'aux plus intimes -, une fois couchées sur clavier, ne seront plus jamais oubliées.
La normalisation simplifie le traitement automatisé des profils, des comportements et des pensées.

\paragraph{} À sa manière, chaque service concentre son activité sur un aspect de la vie de ses utilisateurs. C'est avant 
tout une nécessité stratégique pour ces entreprises qui doivent jouer sur leur avantage différenciatif pour se constituer
une base utilisateur. Ainsi, Une même personne n'aura pas les mêmes usages d'un réseau social "généraliste" comme Facebook, 
d'une plateforme de micro-blogging comme Twitter ou d'un service de partage de photos et vidéos comme Instagram.

\paragraph{} Cette diversité des usages ne change au final pas le sujet : Moi. Mais elle nous permet de l'exprimer différement, 
de nous dévoiler un peu plus. Là où Facebook va permettre de définir le profil d'une personne dans ses grandes lignes, 
Twitter pourra être utilisé pour déduire l'humeur, les opinions, la personnalité d'un individu, et Instagram identifier
ses loisirs \cite{SocialMedia4}. Un usage simple des moteurs de recherche permettra aussi de découvrir les implications
de cette personne dans de nombreux domaines bien qu'elle n'en fasse potentiellement pas part sur les réseaux ! C'est cet 
amas, cette masse de données hétérogènes, qui constitue un profil. Un instantané de notre être... ou de ce que l'on souhaite
en montrer.

\paragraph{} Dans le monde professionnel, il n'est pas rare d'être contacté \guillemotleft sur profil\guillemotright,
car \guillemotleft notre profil correspond\guillemotright à un besoin quelconque. Inversement, il est possible qu'un
entretien d'embauche n'ait jamais de suite car le potentiel futur employeur sera allé se renseigner sur les activités, 
opinions et autres loisirs d'un candidat et y aura trouvé un élément jugé rédhibitoire. C'est bien là l'exemple de la 
recherche d'un \emph{profil type} plutôt que d'un \emph{individu}, d'une illusion de perfection plutôt que du ardu travail
de recherche de la pierre précieuse.

\paragraph{} Si l'on met ce constat en relation avec le développement des Smart Cities, on obtient un résultat
effrayant. Que cela signifierait-il de vivre dans une agglomération ayant accès, à travers les réseaux et services, à une
base de données maintenant à jour les profils de ses habitants ? Ce sont souvent les avantages qui sont mis en avant : 
proposition de produits personnalisés, services sur mesure, évolution et amélioration continue de notre environnement...
Mais aussi accès instantané à l'ensemble de nos réflexions, opinions, goûts, et humeur du moment. Une vie publique sur 
mesure en échange de notre vie privée.

\paragraph{} Il est aussi intéressant de souligner le fait que le \emph{sur-profilage} a un effet de bord à la fois sur
la supposée compréhension que le service a de son utilisateur et sur les pratiques de l'utilisateur lui-même vis-à-vis 
de ce service. En effet, a trop vouloir en cerner les préférences, il y a un risque d'enfermer l'utilisateur dans des
habitudes de consommation de produits ou services plus ou moins identiques, sans qu'il ne cherche ou ne lui soit proposé
quelque chose de nouveau ou - pire ! - de différent. Ce biais est à la fois dommageable pour le service, qui sabotte
\emph{de facto} sa propre diversification de sources de revenus potentiels, et pour l'utilisateur qui, s'il ne s'en 
aperçoit pas, se laisse \emph{habituer} à consommer certains produits ou services et développera, si ce n'est de la 
dépendance, une attente marquée par ses habitudes au sein même du service, mais aussi au dehors.

\paragraph{} Finalement, il est triste de constater que l'évolution de nombreux services depuis ces dix dernières années
tend vers une accentuation de nos caractères égotiques au détriment d'un honnête vivre ensemble.