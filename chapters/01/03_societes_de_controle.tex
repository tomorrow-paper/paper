\section{Les sociétés de contrôle}
\paragraph{Références} \cite{Huxley0} \cite{Orwell0} \cite{TechnoSocio1}

\paragraph{Pourquoi accepter ?} La question de l'acceptation est délicate à aborder, car
très personnelle par certains aspects. Tout d'abord, est-on conscient d'\emph{accepter} ?
Avons-nous conscience de ce qu'impliquent les \emph{End User License Agreement} (EULA) que
nous nous empressons d'accepter sans les lire pour pouvoir profiter d'un nouveau service ?
Avons-nous conscience que, de part notre participation et notre utilisation de ces services
- quels qu'ils soient ! - nous \emph{acceptons} leur existence et leur mode de fonctionnement ?

\paragraph{Une portée personnelle} Se savoir surveillé restreint-il les instincts pervers ?
Ou au contraire cela les désinhibe-t-il ? Il convient avant tout de définir ce que l'on
entend par \emph{instinct}. La surveillance peut-elle vraiment anéantir toute perversion ?
Parallèle possible avec les radars routiers automatiques : ont-ils fait disparaitre les
infractions par excès de vitesse ? Le poblème de la perversion est qu'elle fait partie
intégrante de nous-même, quelle qu'elle soit. Vaudrait-il mieux alors vivre dans une société
où sévissent une dizaine d'êtres extrêmement pervers, ou dans une société où la perversion
est \emph{lissée}, \emph{amortie}, \emph{normalisée}, présente chez tous à un degré infime...
mais toujours présente ? La prévision (d'un crime, par exemple), amènerait-elle à changer
son comportement ?