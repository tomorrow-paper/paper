\chapter{Les Technosociétés}
\paragraph{Objet} Réflexion sociétale
\paragraph{Technique} Récolte automatisée des données personnelles
\paragraph{Références}
\cite{Damasio0}
\cite{Damasio1}
\cite{Damasio2}
\cite{Deleuze0}
\cite{Foucault0}
\cite{Huxley0}
\cite{Klein0}
\cite{Marx0}
\cite{Marx1}
\cite{Moore0}
\cite{Negri0}
\cite{Nietzsche0}
\cite{Orwell0}
\cite{Pieces0}
\cite{Rabhi0}
\cite{Rabhi1}
\cite{Rufin0}
\cite{Arte0}
\cite{GhostInTheShell}
\cite{Gunnm}
\cite{PsychoPass}

\paragraph{} Les technologies font aujourd'hui partie de notre quotidien. Du lever au coucher du soleil, nombreux sont
ceux qui n'agissent plus : ils \emph{ré}agissent. Ils abandonnent leur puissance, c'est-à-dire la capacité à déployer
l'action par soi-même, au profit d'un pouvoir, la capacité de déléguer l'action, conféré par des outils qu'ils ne 
souhaitent pas comprendre.

\paragraph{} Agissant aujourd'hui plus que jamais comme une extension de nous-même, les technologies nous permettent de
réinventer notre rapport au monde et notre rapport aux autres, mais aussi - voire surtout - notre rapport à \emph{soi}.
\cite{Damasio2}

\paragraph{} Mais il n'en a pas toujours été ainsi : ce n'est qu'au court des dernières décennies que nôtre dépendance
- \emph{nôtre} en tant que société et non pas que seuls individus - s'est réellement accentuée.

\section{De nouveaux modèles de sociétés}

\paragraph{} Nos sociétés ont toujours été avides de \emph{contrôle}, et cela indépendamment du régime politique en 
place. 

\paragraph{} Mais qu'entendons-nous exactement par là ? Le contrôle, c'est l'\guillemotleft Action, fait de contrôler
quelque chose, un groupe, d'exercer sur eux un \emph{pouvoir}.\guillemotright. \cite{Controle0} \emph{Contrôler} est
issu de l'ancien français \emph{contreroller} (XIII\up{ème} siècle), signifiant \guillemotleft Vérifier des comptes, des
écritures d'un registre à l'aide d'un second registre.\guillemotright. \cite{Controle1} Dans \emph{contre-roller}, le 
\emph{rolle} - issu du latin \emph{rotulus} - désigne un \guillemotleft Registre\guillemotright  et évoluera pour devenir
notre \emph{rôle}, \guillemotleft Action ou influence exercée par quelqu'un\guillemotright. \emph{Contrôler}, c'est donc
subordonner un pouvoir à un autre, et le \emph{contrôle} possède dans notre français moderne le sens figuré de \guillemotleft
surveillance\guillemotright.

\paragraph{} \guillemotleft La bourgeoisie a subordonné l'Orient à l'Occident\guillemotright  selon Marx et Engels,
mais \guillemotleft Le prolétariat utilisera sa domination politique pour arracher peu à peu tout le capital à la
bourgeoisie, pour centraliser tous les instruments de production entre les mains de l'État, c'est-à-dire du prolétariat
organisé en classe dominante\guillemotright. \cite{Marx1} Il apparaît ici clairement que leur vision
n'est qu'un énième renversement de la bascule des pouvoirs, sans déplacement du rapport de force. Pire que cela, la mise
en place d'un "État prolétaire totalitaire" vise à faire disparaître l'individu (le \emph{Soi}) au profit d'un \emph{Nous}
qui se ferme aux différences : homogénéisation de la classe dominante, refus - voire peur - de l'autre, ce schéma semble
se représenter de manière cyclique à travers les âges.

\paragraph{} L'analogie du modèle carcéral de Foucault souligne elle aussi ces tendances extrêmes.
En effet, les exécutions et supplices publiques étaient un moyen de corriger le tort causé au souverain par le crime ou
délit ; ce dernier exprimait alors une possession pleine et entière de l'individu, et par extension des spectateurs. 
Le contrôle exercé était alors physique sur l'un et psychologique sur les autres, qui craignaient de se retrouver un jour
dans la même fâcheuse posture. Vinrent ensuite les travaux de force, le bagne, dont l'objectif n'était plus d'infliger
une souffrance directe mais d'affirmer une emprise plus entière sur les capacités physiques et mentales des condamnés.
Finalement, on cessera d'avoir recours au supplice pour privilégier l'enfermement. Il n'est pas alors question d'une
pudeur excessive à l'idée de punir, mais plutôt de resituer l'objectif réel de la punition.
\guillemotleft L'âme succède au corps comme objet de l'expiation des crimes.\guillemotright  \cite{Foucault0}.

\paragraph{} Des supplices féodaux à l'enfermement à perpétuité, nos sociétés ont fait évoluer de manière perverse les
mécanismes de contrôle. Le développement de l'architecture panoptique en est un excellent exemple. Imaginé à la fin du
XVIII\up{ème} siècle par Jérémy Bentham \cite{Panoptique1}, le principe du panoptique est de permettre à un individu,
situé dans une tour centrale, de pouvoir surveiller toutes les personnes situées autour de lui dans des cellules traversées
de part en part par la lumière, créant ainsi un contre-jour l'empêchant d'être lui-même vu depuis lesdites cellules.
La définition du dictionnaire Larousse est la suivante : \guillemotleft Panoptique \emph{adj.} : Se dit d'un bâtiment (pénitentiaire,
\emph{hospitalier, etc.}) dont, d'un pont d'observation interne, on peut embrasser du regard tout l'intérieur.\guillemotright
\cite{Panoptique0} Car l'architecture panoptique n'a pas uniquement été pensée pour les milieux carcéraux, mais pour tous
types d'usages nécessitant une quelconque surveillance d'autrui : \guillemotleft [...] un surveillant dans la
tour centrale, et dans chaque cellule d'enfermer un fou, un malade, un condamné, un ouvrier ou un écolier.\guillemotright
\cite{Panoptique2} Son frère Samuel Bentham aurait lui-même mis en application ce modèle architectural... dans un atelier
industriel en Russie, permettant au contremaître de surveiller constamment le travail de ses ouvriers.

\begin{figure}[ht]
    \centering
    \includegraphics[width=350px]{chapters/01/images/panoptique.jpg}
    \caption{\label{panoptique}\emph{Stateville Correctional Center}, Illinois, États-Unis : Prison construite sur le modèle panoptique.}
\end{figure}

\paragraph{} On retrouvera le panoptique dans la littérature de science fiction et d'anticipation du XX\up{ème} siècle.
Dans \emph{La Zone du Dehors} \cite{Damasio0}, Alain Damasio décrit une société désindividualisante qui, à la place
de noms, attribue des matricules changants à des citoyens normés. De manière à assurer \emph{préventivement} l'ordre et la
sécurité, des tours panoptiques aux parois de verre sans tain surplombent la cité et sont accessibles à tous pour épier
et reporter en toute impunité. Cela tend alors à développer chez chacun le sentiment latent et insidieux d'être observé, 
constamment sous surveillance.

\paragraph{} C'est justement sur cet effet que vont reposer, selon Michel Foucault, les nouveaux modèles disciplinaires.
\guillemotleft Le vrai effet du Panopticon, c'est d'être tel que, même lorsqu'il n'y a personne, l'individu dans sa
cellule, non seulement se croie, mais se sache observé, qu'il ait l'expérience constante d'être dans un état de visibilité
pour le regard\guillemotright. \cite{Foucault0} Dès lors, à quoi bon recourir à la violence ? Chacun se sachant, à tort
ou à raison, observé continuellement agira en conséquence. Mais Foucault pousse plus loin son raisonnement, comparant la
panoptique pénitentiaire à un système de documentation des individus. Le geôlier peut alors prélever sans interruption
des informations et observations à propos des détenus, les analysant, les catégorisant, dans le but de \guillemotleft 
[Faire] de la peine rendue nécessaire par l'infraction une modification du détenu, utile pour la société.\guillemotright

\paragraph{} Si l'on repense à la volonté initiale de Bentham d'appliquer le modèle panoptique à toutes sortes d'institutions
(universités, hôpitaux, ...), on assiste alors à l'objectification de l'individu, qui n'est plus considéré que comme le 
n\oe{}ud du complexe maillage qu'est la \emph{Société}, entité transcendentale se nourrissant de ses - ou plutôt de 
\emph{ces} ? - individus pour évoluer, pour le meilleur comme pour le pire.

\paragraph{} Mais de quoi les sociétés sont-elles consitutées sinon d'Hommes ? De là, ce sont bien les aspirations et 
motivations de ceux qui les composent qu'elles reflètent à l'identique, et c'est justement cette volonté de contrôle que
la technologie viendra par la suite outiller. Nous sommes bien loin de l'Oasis de Pierre Rabhi, pleine de paix et de
simplicité \cite{Rabhi1}. Nous allons donc d'abord nous intéresser à l'origine du Réseau, épine dorsale de notre Moi 
numérique, sans laquelle aucun des services que nous utilisons au quotidien n'aurait pu voir le jour.


\paragraph{Historique}

\paragraph{} Les premières recherches ayant pour objectif la mise en place d'un réseau capable de résister à une frappe
nucléaire massive datent de 1957 \cite{Internet0}, quand le \emph{United States Department of Defense} (DoD) prend la
décision de fonder l'\emph{Advanced Research Projects Agency} (ARPA), un groupe scientifique ayant pour mission de concevoir
des innovations technologiques pour l'armée. Épaulés par les chercheurs de la \emph{RAND Corporation} à partir de 1962,
leurs travaux aboutiront un an plus tard sur l'ébauche d'un réseau décentralisé pouvant continuer à fonctionner dans le
cas où une ou plusieurs machines le composant viendraient à s'arrêter. L'idée d'un système décentralisé vient de Paul Baran,
qui inventa avec Donald Davies la communication de données par paquets, ajoutant ainsi à la résilience du système : le réseau
formant un maillage anarchique, chaque paquet pourra emprunter la route la plus courte possible afin de parvenir à sa 
destination, et aura la capacité de patienter durant un laps de temps maximal prédéfini s'il se trouvait dans l'incapacité
de la joindre.

\paragraph{} Le projet fut cependant refusé par l'armée, et il faudra attendre 1969, 6 ans plus tard, pour que celui-ci
se concrétise sous le nom d'\emph{Arpanet}. \emph{Bolt Beranek and Newman Inc.} (BBN) mettra en place, sur commande de
l'ARPA, un réseau reliant quatre des grands centres universitaires américains en utilisant le \emph{Network Control Protocol}
(NCP), sur des lignes pouvant atteindre 50 kilobits par seconde. Les universités concernées sont l'Université de Californie
à Los Angeles (UCLA), l'Institut de Recherche de Stanford (SRI), l'Université de Californie à Santa Barbara (UCSB) et
l'Université de l'Utah.

\begin{figure}[ht]
    \centering
    \includegraphics[width=350px]{chapters/01/images/arpanet_map.jpg}
    \caption{\label{arpanet}Carte du réseau Arpanet, mars 1977.}
\end{figure}

\paragraph{} Il faudra attendre encore deux avancées majeures pour que ce réseau soit utilisable par le grand public :
le protocole TCP/IP, et le DNS.

\paragraph{TCP/IP} La suite TCP/IP, nommée après les deux premiers protocoles qui la composent (le \emph{Transmission Control
Protocol} et l'\emph{Internet Protocol}), est développée en 1973 par Vinton Cerf et Bob Kahn. Son modèle est constitué de
quatre couches prenant en charge la transmission de données : les couches hautes manipulent les données abstraites
présentées à l'utilisateur, tandis que les couches basses permettent la représentation de ces données sur un médium
physique (aujourd'hui : connecteurs 8P8C - abusivement appelés RJ45 -, fibres optiques, ...). Adopté pour l'Arpanet en
1976 sur décision du DoD, il faudra attendre 1983 pour sa mise en place effective.

\paragraph{} Il est important aujourd'hui de bien faire la distinction entre le modèle TCP/IP et le modèle OSI 
(\emph{Open Systems Interconnection}, publié en 1984). Ce dernier, plus rigoureux, est composé de 7 couches et ne
propose pas d'interopérabilité à proprement parler avec le modèle TCP/IP. Il est utilisé presque exclusivement par des
équipements réseau.

\begin{figure}[ht]
    \centering
    \includegraphics[width=350px]{chapters/01/images/tcpip_dataflow.png}
    \caption{\label{tcpip_dataflow}Flux de données du protocole TCP/IP.}
\end{figure}

\paragraph{DNS} Le \emph{Domain Name System} (DNS) est la solution qui a été trouvée pour résoudre les problèmes survenus
suite à l'augmentation de la popularité du réseau Arpanet. En effet, de manière à ce que les différentes machines du 
réseau puissent communiquer entre elles, il était nécessaire qu'elles maintiennent à jour un fichier \emph{hosts.txt}
permettant d'effectuer la conversion entre un nom d'hôte et l'adresse correspondante sur le réseau. En réalité, ce fichier
était maintenu à jour par le \emph{Network Information Center} (NIC) de l'institut de recherche de Stanford, qui centralisait
les modifications qui lui étaient transmises par les différents administrateurs du système et qui s'occupait de redistribuer
périodiquement le fichier mis à jour. Cependant, avec l'augmentation progressive du traffic, le NIC se retrouva dans
l'incapacité de supporter la charge réseau et des problèmes de collision de nom survinrent de plus en plus fréquemment.

\paragraph{} Jon Postel, Paul Mockapetris et Craig Partridge rédigèrent en 1983 les spécifications de ce qui devait 
devenir l'un des piliers d'Internet tel que nous le connaissons aujourd'hui : le DNS. Il s'agit d'une base de données
distribuée stockant les noms de domaines, et divisée en zones. Chacune de ces zones était prise en charge par un ou plusieurs
serveurs de noms (\emph{Name Servers}) répondant aux requêtes des résolveurs (\emph{resolvers}), petits programmes se
chargeant de transformer un nom de domaine en adresse. En 1984, les suffixes DNS, ou \emph{Top Level Domains} (TLD),
font leur apparition. Disposant chacun de leur propre zone DNS, ils permettront dans un premier temps de regrouper les
noms de domaine par fonction : \emph{.com} pour les sites commerciaux, \emph{.gov} pour les sites gouvernementaux des
États-Unis, etc. Avec le temps, certaines de ces particularités furent dépréciés (les sites en \emph{.com}
ne fournissent plus obligatoirement des services d'ordre commerciaux), et l'on vit apparaître les TLD nationaux (\emph{.fr},
\emph{.us}, \emph{.uk}, ...) dont l'obtention est soumise aux lois du pays concerné.

\paragraph{} Une fois ces systèmes mis en place, le réseau qui n'était à l'origine destiné qu'à des fins de recherche
militaire avait atteint un seuil majeur d'évolutivité. L'invention du "Web" est historiquement située en 1989 et
attribuée à Sir Timothy John Berners-Lee, alors chercheur au \emph{Conseil Européen pour la Recherche Nucléaire} (CERN)
de Genève, et à son texte \guillemotleft Information Management: A Proposal\guillemotright \cite{Internet1}. Sa volonté
est alors de mettre en place un système d'information global pour la recherche, à l'image d'Arpanet - mais accessible par tous.
Il est l'auteur d'\emph{httpd}, le premier serveur HTTP, et de \emph{WWW} (\emph{World Wide Web}), le premier client.
Il préside depuis 1994 le \emph{World Wide Web Consortium} (W3C), organisme qu'il a lui-même fondé et chargé de mettre
au point les nouveaux standards des technologies Web.

\paragraph{} Toutes les briques étaient en place pour l'adoption par le grand public : les technologies, réseaux comme
logicielles, étaient fin prêtes. Pour un rapide rappel de la chronologie évoquée ici, rendez-vous dans à l'annexe
\ref{chronology} en page \pageref{chronology}.

\paragraph{} Nous assistons depuis à une démocratisation à marche forcée des technologies dans notre quotidien. 
Interconnectés, nous le sommes de plus en plus : à tel point que les industries commes les pouvoirs politiques ne
peuvent plus ignorer ni l'usage ni les réactions des réseaux. Un battement d'ailes de papillon peut provoquer de nos
jours bien plus qu'une simple tornade.
\section{L'irruption des technologies}

\paragraph{Collecter : hier et aujourd'hui}

\paragraph{} Le terme de \emph{collecte} n'est pas récent ; en effet il nous viendrait vraisemblablement
du XIVe siècle et du terme \emph{collete} en latin, qui signifie la \guillemotleft levée des impositions\guillemotright,
soit la période pendant laquelle un collecteur était en fonction. Cette définition relève de plusieurs notions
qui nous intéressent plus particulièrement.

\paragraph{} Tout d'abord, la collecte c'est la levée d'un impôt. C'est donc que le collecté \emph{doit}
quelque chose au collecteur, pour une \emph{raison} particulière : mise à disposition de biens, services,
protection... La collecte est donc un échange entre deux entités. Cet échange est avant tout un contrat de
pouvoir : contre tel service, en échange de tel accès, de telle reconnaissance, le collecté accepte de
léguer une part de ses biens, de ce qu'il est ou ce qu'il a. Ce qui n'est pas le cas du collecteur qui est
dans une situation de \emph{prêt}, car il ne pourrait pas, en effet, mettre à disposition un service dont
il ne contrôle pas le flux. Cela l'empêcherait de pratiquer sa \emph{condition de collecteur}, n'étant pas en
\emph{situation de force} par rapport aux collectés.

\paragraph{} Il y a donc un rapport de pouvoir entre collecteur et collecté. Le collecteur s'apparente à un n\oe{}ud
parent, qui exerce le même contrôle sur tous ses noeuds enfants, qui eux-mêmes ne peuvent finalement qu'accepter
le pouvoir émis par ce dernier, à partir du moment où il n'existe pas d'autre collecteur offrant le \emph{même
service} pour une \emph{peine moindre}. On reconnait là une première forme de centralisation des données.
Au départ, la collecte concerne évidemment les revenus, l'argent d'un foyer ou d'un particulier. Car cet
argent est le pouvoir de vie (ou de \emph{survie}) d'un Homme, il doit accepter d'en laisser, d'en léguer un peu
s'il souhaite avoir accès à un bien. Mais l'argent étant assez mal réparti au sein des sociétés \cite{Richesses0},
tous ne peuvent se permettre de payer. Ce sont alors d'autres biens qui sont collectés : récoltes, créations
artisanales... C'est le produit de la \emph{technique} de l'Homme qui est récolté. Ce sont son \emph{temps} et
ses \emph{capacités techniques} qui lui permettent de payer.

\paragraph{} Néanmoins, ces collectes demandent une contribution matérielle de la part du collecté, et ce n'est pas
nécessairement le cas. Un exemple ancien serait celui des citoyens, en Grèce antique, qui devaient être répertoriés
dans des tables afin de témoigner de leur condition de citoyen. A titre exceptionnel, des femmes ou des esclaves
devenaient parfois citoyen, et la présence de leur nom sur une tablette leur permettait de l'affirmer et d'être
reconnu comme tel. Un parallèle peut être fait avec l'exemple plus récent des nationalités et des cartes d'identité :
avoir un passeport ou une carte d'identité, c'est être citoyen d'un pays. C'est donc bénéficier des \emph{lois} de ce
pays quelque soit le cadre physique (ou presque). Dans les deux cas, le citoyen ne paie pas son recensement par de l'argent,
des récoltes ou autres. En revanche, il est \emph{de son devoir} d'être recensé, sinon il ne pourrait bénéficier des 
conséquences de son recensement. On peut alors statuer qu'il y a des récoltes de données qui sont \emph{indirectement
obligatoires} pour l'Homme bien qu'elles ne soient pas \emph{matériellement coûteuses}.

\paragraph{} La collecte est donc, qu'elle soit directe ou indirecte, \emph{imposée}. Dans nos exemples précédents,
la collecte est toujours \emph{acceptée} et \emph{consciente}. Acceptée tout d'abord car il n'y a finalement pas le
choix : si je souhaite bénéficier de quelque chose, je dois pouvoir céder une contrepartie. C'est la nature même des
échanges humains - je ne suis prêt à te donner ceci qu'\emph{en échange de} cela. Il y a donc une interaction directe 
: deux humains intéragissent pour parvenir à un accord. Mais ce n'est justement que parce que cette collecte met en
jeu deux humains qu'elle peut être qualifiée de \emph{consciente} - parce que le collecté est capable de se dire à
un instant précis qu'\emph{il est effectivement en train de céder la contrepartie} à quelque chose. Ce n'est pas
toujours le cas et c'est pourquoi nous devons distinguer deux formes particulières d'acceptation de la collecte :
\emph{consciente} et \emph{inconsciente}.   

\paragraph{Collectes et consciences} \label{collect_data_conscious}

\paragraph{} Avant même de parler de conscience de la collecte, il est de bon ton de rappeler,
comme le précise C.G. Jung, qu'\guillemotleft Il est assez stérile d'étiqueter les gens et de les presser dans
des catégories\guillemotright. Par ici, on entend que l'\emph{expérience} même de la conscience est différente
pour tout un chacun. On parlera donc de "collecte affirmée" pour ce qui est conscient, et de "collecte non précisée"
pour ce qui est inconscient. Tout le monde n'est pas conscient que les publicités que l'on voit sur internet sont
ciblées ; et même si les utilisateurs y sont de plus en plus attentifs, il n'est pas indiqué à côté de chaque
publicité "Cette pub pour tel achat est présente car nous avons collecté vos dernières recherches et que plus d'un
quart concerne ce sujet ou des sujets proches". 

\paragraph{} Tout ce qui est inconscient est plus évident à aborder en premier lieu pour une raison simple : tout est
inconscient par définition. Prendre conscience de, c'est \emph{considérer} quelque chose, y porter \emph{son attention}.
Or, par nature, l'Homme ne porte attention qu'à une chose à la fois, et certainement pas à l'ensemble de son
environnement. Toute donnée doit donc, dans l'absolu, être collectée de façon inconsciente : de l'heure du réveil au
temps mis pour se préparer, l'itinéraire choisi pour aller au travail ou faire des courses, les lieux et les personnes
qui ont été fréquentées... L'ensemble des données générées au quotidien par tout être humain est collectable par
définition. Les questions qui se posent donc sont : comment, et quelles données sont pertinentes ?

\paragraph{} Afin d'apporter une réponse à la première question il est important de s'arrêter un instant sur le fait
suivant : lorsque l'on rend la collecte \emph{nécessaire pour le collecté} - comme dans l'exemple précédent sur la
citoyenneté - elle n'est plus \emph{inconsciente}, car elle est \emph{affirmée}. La nécessité de la collecte porte
l'attention du collecté sur cette dernière, qui devient alors par définition \emph{consciente}. Cela
implique donc qu'une collecte est \emph{inconsciente} uniquement si elle \emph{ne créé par de nécessité}
chez le collecté. On se rend alors immédiatement compte que c'est l'arrivée de nouvelles techniques et de la
technologie qui permet une collecte de données \emph{inconsciente}. En s'infiltrant dans les interstices de nos vies -
smartphone, ordinateur, internet, transports, paiements - la technologie \emph{traque tout} et \emph{tout le temps},
sans que nous ne nous en aperçevions. Une évolution importante de nos jours, et que nous détaillerons par la suite, est
celle de l'IoT (Internet of Things) et des objets connectés en général. Leur utilisation permet une collecte plus
poussée mais surtout \emph{plus proche} de l'utilisateur, et donc une récolte de données bien plus \emph{précise}. Or
nous verrons lorsque nous aborderons le thème du Machine Learning qu'il existe un lien fort entre la précision d'une
donnée et la qualité du service qui pourra être fourni à un utilisateur (voir \ref{select_data_ml}).

\paragraph{} Puisque ces données sont partout, et qu'elles vont être de plus en plus récupérées, il est devenu
une question importante que celle du \emph{stockage} de ces données. Afin d'optimiser ce stockage, et avant même de
penser aux techniques physiques utilisées pour garder des données, il faut effectuer un tri dans ce qui sera récupéré.
Pourtant, ce tri est aujourd'hui davantage effectué une fois que la donnée est récupérée. En effet, afin de déterminer
si une donnée est pertinente, il suffit de se demander si elle apporte des informations suffisantes à la récupération
d'un bénéfice futur. C'est le principe même de \emph{collecte} : le collecté consomme et le collecteur en tire \emph{
une valeur}, directe ou indirecte. C'est donc bien qu'il y a une question de \emph{monétisation} de la donnée. La
problématique n'est plus donc de savoir quoi récupérer mais plutôt de comment exploiter ce qui à été récupéré.

\paragraph{} Dès lors que nous parlons de création de valeur, cela replace le contexte de la collecte dans un échange
\emph{conscient}, ou plutôt \emph{affirmé}, de la donnée. Il faut au départ que le collecteur produise un \emph{service},
qu'il existe un \emph{contrat} entre collecteur et collecté pour que la création de données ait lieu. Une matérialisation
simple et actuelle de cela se retrouve lorsqu'un utilisateur installe une application sur son smartphone - il est
alors invité à accepter ou non que l'application ait à son tour accès à un certain nombre de choses (medias, gallerie,
données personnelles, agenda...) sur le smartphone. C'est la partie \emph{consciente} du choix. Une fois qu'elle est
\emph{acceptée}, la partie \emph{inconsciente est alors acceptée également par défaut}. En effet le collecté accepte que
l'utilisation qu'il va faire de l'application soit collectée \emph{de facto}, par la \emph{nécessité d'accès à un service}.

\paragraph{Prototype}
\label{prototype_recuperator}

\paragraph{} Pour étayer notre propos, un prototype servira de fil rouge tout au long de votre lecture : \lstinline{Tomorrow}
(\url{https://github.com/tomorrow-paper}). Nous détaillerons à la fois les objectifs remplis par chacun de ses modules
ainsi que les choix techniques que nous avons effectués.

\paragraph{} Nous avons choisi le \lstinline{Rust} \cite{Rust0} comme principal langage de programmation, car il allie une syntaxe
de haut niveau moderne avec tous les avantages en terme de performance apportés par des exécutables natifs compilés.
Multi-paradigmes, l'une des grandes forces de ce langage est sa gestion extrêmement stricte de la mémoire, introduisant
des concepts nouveaux comme l'\emph{ownership} (propriété) et le \emph{borrowing} (emprunt) d'espaces et de références 
mémoires permettant d'assurer que le code produit est exempt de fuites, \emph{data races}, problèmes de parallélisation...
Son gestionnaire de dépendances, \lstinline{cargo}, ainsi que son riche écosystème de \emph{crates} (bibliothèques) sont
à même de réconcilier les plus réfractaires avec les langages de programmation système.

\paragraph{} Pour accompagner ce premier chapitre, nous avons souhaité mettre en place un service permettant d'aggréger
des données depuis n'importe quelle source pour ensuite pouvoir les consommer de manière normalisée.
Nous nous intéresserons ici à la récupération de données personnelles, et avons donc pour cela restreint notre champs
d'action à quelques services pertinents :

\begin{itemize}
    \item Google : récupération des suggestions de pages liées à la personne concernée par le
    premier moteur de recherche mondial (\url{https://encrypted.google.com/search?q={SEARCH_TERMS}}) ;
    \item Google Maps API : récupération des coordonnées géographiques associées à une adresse
    (\url{https://maps.google.com/maps/api/geocode/json?address={ADDRESS}}) ;
    \item Facebook : recherche et récupération des données publiques des profils utilisateurs
    (\url{https://www.facebook.com/public/{PEOPLE}}).
\end{itemize}

\paragraph{} De manière à préserver la séparation des responsabilités des composants de notre prototype, plusieurs modules
ont été développés : 

\begin{itemize}
    \item \lstinline{tomorrow-core} (\url{https://github.com/tomorrow-paper/tomorrow-core}) : contient la logique et les
    structures de données communes à tous les composants du prototype (gestion d'erreur, type résultat, ...) ;
    \item \lstinline{tomorrow-http} (\url{https://github.com/tomorrow-paper/tomorrow-http}) : contient les interfaces et
    implémentations de la logique de requêtage à travers le protocole HTTP.
    \item \lstinline{tomorrow-recuperator} (\url{https://github.com/tomorrow-paper/tomorrow-recuperator}) : contient les interfaces
    de récupération de données.
\end{itemize}

\paragraph{} Ce découpage n'est pas anodin. Il nous permet par exemple de développer, pour chaque service cible, un module
dédié implémentant les interfaces définies dans \lstinline{tomorrow-recuperator}  :

\begin{lstlisting}
use tomorrow_core::Result;

pub trait Request {}
pub trait Response {}

pub trait Recuperator<Req, Res> where Req: Request, Res: Response {
    fn compute(&self, request: Req) -> Result<Res>;
}\end{lstlisting}

\paragraph{} Ces différents \emph{récupérateurs} sont ensuite utilisés pour constituer un composant à plus haut niveau
d'abstraction : \lstinline{tomorrow-api} (\url{https://github.com/tomorrow-paper/tomorrow-api}). API REST produisant des
documents JSON, elle sert de façade pour d'éventuelles applications ou un usage à des fins de recherche par un être humain. 
Pour cette première version de notre prototype, aucun \emph{état} ni aucune donnée ne sont persistés. Les informations 
sont lues à un instant \emph{T} quand l'utilisateur en fait la demande et lui sont immédiatement renvoyées par l'API.
Les \emph{endpoints} actuels sont les suivants (tous les modules peuvent être trouvés sur notre dépôt) :

\begin{itemize}
    \item \url{/services/maps/{ADDRESS}} : Google Maps grâce au module \lstinline{recuperator-google-maps} ;
    \item \url{/services/search/{QUERY}} : Google Search grâce au module \lstinline{recuperator-google} ;
    \item \url{/services/facebook/public/{PEOPLE}} : Facebook Public Search grâce au module \lstinline{recuperator-facebook}.
\end{itemize}

\paragraph{} Nos objectifs en développant ce service étaient triples. Tout d'abord, il se devait d'être agnostique en terme
d'utilisation : une API REST ne repose que sur le protocole HTTP, et le format JSON est aujourd'hui un standard en terme
d'encodage d'information. Ensuite, nous souhaitions l'architecturer de manière évolutive : pour récupérer les données
depuis un nouveau service, il suffit de créer un module implémentant les interfaces \lstinline{tomorrow-recuperator} et
de l'intégrer à l'API derrière un nouvel \emph{endpoint}. Enfin nous espérons qu'il vous aidera à prendre conscience que
rien n'empêche l'exploitation d'une donnée accessible publiquement sur Internet de nos jours. Car peu importe le nombre
de services différents que nous pourrions intégrer à ce prototype : c'est sa nature même qui doit vous interpeler.
\section{Les sociétés de contrôle}

\paragraph{Une portée personnelle}

\paragraph{} L'utilisation d'un service est de nos jours synonyme de production et donc collecte de données ; ne dit-on
pas que si un service est gratuit, c'est que ses utilisateurs en sont les produits ? Dès lors nous ne sommes plus maîtres
des informations que nous publions, qui peuvent être vendues ou récupérées à des fins diverses : publicité et marketing
ciblés en sont des exemples bien connus.

\paragraph{} Mais les données ne sont pas uniquement utilisées de manière ciblée. Le développement ces dernières années 
des \emph{Smart Cities} donne naissance à des initiatives impensables auparavant. Ainsi la ville de New York s'est-elle
dôtée d'un tableau de bord \cite{ProgrammableCity1} aggrégeant l'ensemble des sources de données à sa disposition pour 
mettre en exergue de nombreux indicateurs : fluctuation du prix de l'immobilier, évolutions du nombre de plaintes pour 
tapage nocturne par quartier, niveau de saleté dans les rues, état de la circulation aux différentes heures de la 
journée... Toutes ces données dont \emph{nous} sommes la source sont ainsi mises au service de la ville.

\paragraph{} \emph{The Programmable City} \cite{ProgrammableCity0} est un projet proposant de mettre la technologie au
service de l'urbanisme. Une infrastructure et un ensemble de programmes nous permettent de piloter et d'être à l'écoute
de la ville, qui bénéficie au quotidien d'une attention nouvelle : on crée alors un cercle vertueux. Le site
\url{http://DublinDashboard.ie/} est une mise en application de ces concepts pour la ville de Dublin, proposant des
dizaines de sources de données visualisables concernant l'agglomération. D'autres initiatives, comme celle de SideWalk Labs
à Toronto \cite{ProgrammableCity3}, visent à réinventer la ville pour remettre l'humain au c\oe{}ur des processus de
décision en utilisant les nouvelles technologies et les données générées pour améliorer de manière continue les services
qu'elle fournit.

\begin{figure}[h]
    \centering
    \includegraphics[width=300px]{chapters/01/images/programmable_city.png}
    \caption{\label{programmable_city} Le cycle de transduction des \emph{Programmable Cities}}
\end{figure}

\paragraph{} Une donnée, ce n'est rien de plus que de l'information numérisée. Là où ces bases de données publiques
émanant de structures privées ne nous étonnent en rien - car nous sommes \emph{habitués} à les alimenter, comme l'application
Uber Movement permettant d'analyser les trajets effectués via le service Uber -, il est important de comprendre
qu'elles ne diffèrent pas de celles de nos services publics. Ce n'est qu'avec l'adoption progressive de l'\emph{Open
Data} que ces derniers se sont petit à petit ouverts au développement d'applications exploitant toutes sortes de données. 

\paragraph{} Bien évidemment, les \emph{Open Data} et autres sources de données des \emph{Smart Cities} ne sont pas
constituées de données personnelles - c'est à dire d'informations permettant d'identifier de manière directe ou non une 
personne physique \cite{PersonalData0}. Mais cela est-il réellement différent ? Ainsi la société Hitachi a-t-elle développé
en 2015 un système de détection des crimes reposant sur du machine learning \cite{ProgrammableCity2}. Une quantité
astronomique de données de toutes sortes est nécessaire pour alimenter le système, des mouvements de population aux 
antécédents de crimes enregistrés dans les zones sous surveillance. Ne peut-on pas alors craindre de voir une zone 
démographique désertée suite à un incident car son \emph{coefficient de criminalité} aura augmenté, rebutant nouveaux
résidents et jeunes parents à s'y installer ?

\paragraph{} Se savoir surveillé restreint-il les instincts pervers ?
Ou au contraire cela les désinhibe-t-il ? Il convient avant tout de définir ce que l'on
entend par \emph{instinct}. La surveillance peut-elle vraiment anéantir toute perversion ?
Parallèle possible avec les radars routiers automatiques : ont-ils fait disparaitre les
infractions par excès de vitesse ? Le poblème de la perversion est qu'elle fait partie
intégrante de nous-même, quelle qu'elle soit. Vaudrait-il mieux alors vivre dans une société
où sévissent une dizaine d'êtres extrêmement pervers, ou dans une société où la perversion
est \emph{lissée}, \emph{amortie}, \emph{normalisée}, présente chez tous à un degré infime...
mais toujours présente ? La prévision (d'un crime, par exemple), amènerait-elle à changer
son comportement ? Panoptique, surveilleur-surveillé. 


\paragraph{Évolution du partage}

\paragraph{} Évolution du partage : les réseaux sociaux, quelles moeurs ? Catégoriees d'âges ? Sociaux-professionnelles ?
Quelle forme de contrôle incarnée par notre prototype ? À qui est-il destiné ? Pour quelles utilisations ?
La place de l'information/désinformation ?


\paragraph{L'individu technique}

\paragraph{} Comment les modèles de sociétés nous ont-ils amené à réfléchir l'individu en termes technologiques ? 
