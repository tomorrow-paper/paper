\chapter{Les Technosociétés}
\paragraph{Objet} Réflexion sociétale
\paragraph{Technique} Récolte automatisée des données personnelles
\paragraph{Références}
\cite{Damasio:0}
\cite{Damasio:1}
\cite{Deleuze:0}
\cite{Foucault:0}
\cite{Huxley:0}
\cite{Klein:0}
\cite{Marx:0}
\cite{Marx:1}
\cite{Moore:0}
\cite{Negri:0}
\cite{Nietzsche:0}
\cite{Orwell:0}
\cite{Pieces:0}
\cite{Rabhi:0}
\cite{Rufin:0}
\cite{Arte:0}
\cite{GhostInTheShell}
\cite{Gunnm}
\cite{PsychoPass}

\section{De nouveaux modèles de sociétés}
\paragraph{Références} \cite{Marx:0} \cite{Marx:1} \cite{Nietzsche:0}

\paragraph{Ère pré-internet} Analyse des dernières décénnies du XX\up{ème} siècle :
comment les sociétés ont évoluées avec l'arrivée des nouvelles technologies : du
pré-internet à l'ère des réseaux ; jusqu'à aujourd'hui. Quels étaient les objectifs
premiers des ces technos. (la raison de leur développement) ? Comment ont-elles réellement
 été utilisées/ont été amenées à évoluer ? Est-ce réellement une mauvaise chose (n'y
 a-t-il que du mal qui en soit ressorti) ? Est-il pertinent/nécessaire d'effectuer un bref
 historique des technologies \emph{disruptives} (machine de Turing, architecture
 Von Neumann, premiers réseaux, ...) ?

\paragraph{L'individu technique} Comment les modèles de sociétés nous ont-ils amené à
réfléchir l'individu en termes technologiques ?


\section{L'irruption des technologies}
\paragraph{Références} \cite{Damasio:0} \cite{Marx:1} \cite{TechnoSocio:0} \cite{GhostInTheShell}

\paragraph{Collecte des données} Que collecter ? Et surtout, \emph{pourquoi} collecter ?
\begin{itemize}
    \item Argent
    \item Pouvoir : pourquoi et comment est-il exercé ? Parle-t-on d'un pouvoir exercé de
    manière \emph{direct} (pressions, ...), ou \emph{indirect} (pression sociale,
    narcissisme, ...) ?
    \item Prévention : données utilisées à des fins de prévention des crimes, épidémies, ...
    \item Avancées scientifiques : utilisation bénéfique de la masse de données collectées
    pour profiter à la recherche scientifique et aux inovations. Risques de dérives ? Il y
    en a toujours eu : expérimentations sur des êtres humains ; tests nucléaires... 
\end{itemize}


\section{Les sociétés de contrôle}
\paragraph{Références} \cite{Huxley:0} \cite{Orwell:0} \cite{TechnoSocio:1}
\paragraph{} Pourquoi accepter ?
\paragraph{} Se savoir surveillé restreint-il les instincts pervers ?