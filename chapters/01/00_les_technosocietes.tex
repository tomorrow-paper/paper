\chapter{Les Technosociétés}
\paragraph{Objet} Réflexion sociétale
\paragraph{Technique} Récolte automatisée des données personnelles
\paragraph{Références}
\cite{Damasio:0}
\cite{Damasio:1}
\cite{Deleuze:0}
\cite{Foucault:0}
\cite{Huxley:0}
\cite{Klein:0}
\cite{Marx:0}
\cite{Marx:1}
\cite{Moore:0}
\cite{Negri:0}
\cite{Nietzsche:0}
\cite{Orwell:0}
\cite{Pieces:0}
\cite{Rabhi:0}
\cite{Rufin:0}
\cite{Arte:0}
\cite{GhostInTheShell}
\cite{Gunnm}
\cite{PsychoPass}

\section{De nouveaux modèles de sociétés}
\paragraph{Références} \cite{Marx:0} \cite{Marx:1} \cite{Nietzsche:0}

\paragraph{Ère pré-internet} Analyse des dernières décénnies du XX\up{ème} siècle :
comment les sociétés ont évoluées avec l'arrivée des nouvelles technologies : du
pré-internet à l'ère des réseaux ; jusqu'à aujourd'hui. Quels étaient les objectifs
premiers des ces technos. (la raison de leur développement) ? Comment ont-elles réellement
 été utilisées/ont été amenées à évoluer ? Est-ce réellement une mauvaise chose (n'y
 a-t-il que du mal qui en soit ressorti) ? Est-il pertinent/nécessaire d'effectuer un bref
 historique des technologies \emph{disruptives} (machine de Turing, architecture
 Von Neumann, premiers réseaux, ...) ?

\paragraph{L'individu technique} Comment les modèles de sociétés nous ont-ils amené à
réfléchir l'individu en termes technologiques ?

\paragraph{Objectif} L'objectif ici est de démontrer que, suite au développement des technologies, c'est l'Homme
qui génère lui-même des données le concernant, par l'utilisation au quotidien des services (entre autres).
Cela doit être démontré par la mise en place d'un prototype récupérant des données de différents
services (ie facebook, twitter...)


\section{L'irruption des technologies}
\paragraph{Références} \cite{Damasio:0} \cite{Marx:1} \cite{TechnoSocio:0} \cite{GhostInTheShell}

\paragraph{Collecte des données} Que collecter ? Et surtout, \emph{pourquoi} collecter ?
\begin{itemize}
    \item Argent
    \item Pouvoir : pourquoi et comment est-il exercé ? Parle-t-on d'un pouvoir exercé de
    manière \emph{directe} (pressions, ...), ou \emph{indirecte} (pression sociale,
    narcissisme, ...) ?
    \item Prévention : données utilisées à des fins de prévention des crimes, épidémies, ...
    \item Avancées scientifiques : utilisation bénéfique de la masse de données collectées
    pour profiter à la recherche scientifique et aux innovations. Risques de dérives ? Il y
    en a toujours eu : expérimentations sur des êtres humains ; tests nucléaires... 
\end{itemize}


\section{Les sociétés de contrôle}
\paragraph{Références} \cite{Huxley:0} \cite{Orwell:0} \cite{TechnoSocio:1}

\paragraph{Pourquoi accepter ?} La question de l'acceptation est délicate à aborder, car
très personnelle par certains aspects. Tout d'abord, est-on conscient d'\emph{accepter} ?
Avons-nous conscience de ce qu'impliquent les \emph{End User License Agreement} (EULA) que
nous nous empressons d'accepter sans les lire pour pouvoir profiter d'un nouveau service ?
Avons-nous conscience que, de part notre participation et notre utilisation de ces services
- quels qu'ils soient ! - nous \emph{acceptons} leur existence et leur mode de fonctionnement ?

\paragraph{Une portée personnelle} Se savoir surveillé restreint-il les instincts pervers ?
Ou au contraire cela les désinhibe-t-il ? Il convient avant tout de définir ce que l'on
entend par \emph{instinct}. La surveillance peut-elle vraiment anéantir toute perversion ?
Parallèle possible avec les radars routiers automatiques : ont-ils fait disparaitre les
infractions par excès de vitesse ? Le poblème de la perversion est qu'elle fait partie
intégrante de nous-même, quelle qu'elle soit. Vaudrait-il mieux alors vivre dans une société
où sévissent une dizaine d'êtres extrêmement pervers, ou dans une société où la perversion
est \emph{lissée}, \emph{amortie}, \emph{normalisée}, présente chez tous à un degré infime...
mais toujours présente ? La prévision (d'un crime, par exemple), amènerait-elle à changer
son comportement ?