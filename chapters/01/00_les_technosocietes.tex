\chapter{Les Technosociétés}
\paragraph{Objet} Réflexion sociétale
\paragraph{Technique} Récolte automatisée des données personnelles
\paragraph{Références}
\cite{Damasio:0}
\cite{Damasio:1}
\cite{Deleuze:0}
\cite{Foucault:0}
\cite{Huxley:0}
\cite{Klein:0}
\cite{Marx:0}
\cite{Marx:1}
\cite{Moore:0}
\cite{Negri:0}
\cite{Nietzsche:0}
\cite{Orwell:0}
\cite{Pieces:0}
\cite{Rabhi:0}
\cite{GhostInTheShell}
\cite{Gunnm}
\cite{PsychoPass}

\section{De nouveaux modèles de sociétés}
\paragraph{} Pré-internet/réseau, 20ème siècle, ...
\paragraph{} Comment les modèles de sociétés nous ont-ils amené à réfléchir l'individu en termes technologiques ?

\section{L'irruption des technologies}
\paragraph{} Pourquoi collecter ?
\paragraph{} Que collecter ? Argent, pouvoir => pourquoi et comment est-il exercé ? 

\section{Les sociétés de contrôle}
\paragraph{} Pourquoi accepter ?
\paragraph{} Se savoir surveillé restreint-il les instincts pervers ?