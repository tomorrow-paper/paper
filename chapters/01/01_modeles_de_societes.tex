\section{De nouveaux modèles de sociétés}

\paragraph{} Les technologies font aujourd'hui partie de notre quotidien. Du lever au coucher du soleil, nombreux sont
ceux qui n'agissent plus : ils \emph{ré}agissent, abandonnant leur puissance au profit d'un pouvoir conféré par des
outils qu'ils ne souhaitent pas comprendre.

\section*{Brouillon}

\paragraph{Références} \cite{Marx0} \cite{Marx1} \cite{Nietzsche0}

\paragraph{Ère pré-internet} Analyse des dernières décénnies du XX\up{ème} siècle :
comment les sociétés ont évoluées avec l'arrivée des nouvelles technologies : du
pré-internet à l'ère des réseaux ; jusqu'à aujourd'hui. Quels étaient les objectifs
premiers des ces technos. (la raison de leur développement) ? Comment ont-elles réellement
été utilisées/ont été amenées à évoluer ? Est-ce réellement une mauvaise chose (n'y
a-t-il que du mal qui en soit ressorti) ? Est-il pertinent/nécessaire d'effectuer un bref
historique des technologies \emph{disruptives} (machine de Turing, architecture
Von Neumann, premiers réseaux, ...) ?

\paragraph{L'individu technique} Comment les modèles de sociétés nous ont-ils amené à
réfléchir l'individu en termes technologiques ?

\paragraph{Objectif} L'objectif ici est de démontrer que, suite au développement des technologies, c'est l'Homme
qui génère lui-même des données le concernant, par l'utilisation au quotidien des services (entre autres).
Cela doit être démontré par la mise en place d'un prototype récupérant des données de différents
services (ie facebook, twitter...)

\paragraph{Damasio : Très humain plutôt que transhumain}

\begin{quotation}
    L'homme réinvente sans cesse ce qu'il est par la technologie. Nous réinventons notre
    rapport au monde, notre rapport aux autres ; peut être et même surtout notre rapport à soi, 
    notre construction de soi, par la technologie. \cite{Damasio2}
\end{quotation}

Extension de soi - dispositifs tactils et bienveillants autour de nous, qui nous assistent,
nous prolongent, et font de nous de jeunes dieux agiles auxquels le fantôme digital du monde répondrait
; au doigt, et à l'oeil - évidemment.

On y gagne une maitrise, un pouvoir accru sur le monde. Mais n'y perd-on pas quelque chose ?
La "puissance" de Spinoza : celle de vivre, d'agir, d'habiter le monde, de persévérer dans mon être.

Le pouvoir = faire faire - déléguer l'action.
La puissance = la capacité de faire - de déployer l'action par soi-même.

Image de l'oignon technologique, que l'on ne sait plus ni éplucher ni cuisiner à notre
sauce sans risquer de pleurer.

La technologie vient outiller nos paresses. Faciliter nos vies, les fluidifier : on lui
délègue nos efforts, on lui sous-traite nos fatigues. Externalisation des capacités physiques :
voitures, métro, tapis roulant, ... Externalisation des capacités cognitives dans la machine :
mémoire = moteur de recherche, orientation = GPS, organisation = applications de
calendrier/planning - c'est aujourd'hui notre quotidien.

La technologie conjure nos peurs, principalement celles de la solitude et de l'abandon :
le réseau, le portable. Plus jamais SEUL.

La technologie est une eau qui s'infiltre dans tous les interstices, dans tous les blancs,
dans tous les doutes, dans tous les temps morts angoissés de nos vies pour les combler.
On lui demande de pouvoir contrôler notre environnement. Tout bouge sans que rien n'arrive. 

La technologie donne l'espoir de dépasser notre finitude. Le transhumanisme vise à
externaliser dans la technique ce que notre chaire et notre esprit ne sont supposément
pas capables de faire par eux-même.

La technologie ne nous pousse pas à *agir*, mais à *réagir* à des stimulis extérieurs :
notifications, mails, SMS, discussions instantanées, ...

Agir, c'est produire un événement qui libère un peu de vie là où les dispositifs de contrôle
l'étouffe en la gérant. C'est sentir qu'indépendamment de nos outils, nous sommes une puissance.


\section*{Prototype}

\paragraph{} Pour étayer notre propos, un prototype servira de fil rouge tout au
long de votre lecture. Nous nous intéresserons ici à la récupération de données personnelles,
focus de cette première partie. Pour cela, nous avons restreint notre champs d'action à quelques
services pertinents pouvant nous apporter un maximum d'informations :

\begin{itemize}
    \item Facebook : récupération des données publiques des profils utilisateurs ;
    \item Twitter : récupération du profil et des derniers messages de l'utilisateur, jugés
    représentatifs de son \emph{opinion} ;
    \item Instagram : récupération des photos de l'utilisateur ; 
    \item Google Maps API : récupération des coordonnées géographiques associées à une adresse
    (\url{https://maps.google.com/maps/api/geocode/json?address={ADDRESS}}) ;
    \item Google : récupération des suggestions de pages liées à la personne concernée par le
    premier moteur de recherche mondial (\url{https://encrypted.google.com/search?q={SEARCH_TERMS}}) ;
    \item LinkedIn : récupération du profil professionnel de l'utilisateur (\emph{nécessite une clé d'API ?}).
\end{itemize}

\paragraph{} Afin de permettre une aggrégation simple de l'ensemble de ces données, une 
API REST sert de façade pour d'éventuelles applications ou un usage à des fins de recherche
par un être humain. La récupération concertée des informations concernant une personne peut
être effectuée grâce à une requête \lstinline{GET} sur l'URL \url{https://{API_URL}/people/{NAME}}
- la réponse est au format JSON.

\paragraph{} En arrière plan, plusieurs composants entrent en jeu. Pour chacun des services
ci-dessus, un \emph{récupérateur} dédié a la charge de l'obtention et du formatage des données
exposées. Ces différentes informations sont ensuite retravaillées et aggrégées de manière 
à créer un "\emph{profil utilisateur}" le plus complet possible à partir des informations
disponibles publiquement sur internet.

\paragraph{} Pour cette première version de notre prototype, aucun \emph{état} ni aucune
donnée ne sont persistés. Les \emph{endpoints} de notre API sont les suivants :

\begin{itemize}
    \item \url{/people/{NAME}}
    \item \url{/services/facebook/{NAME}}
    \item \url{/services/twitter/{NAME}}
    \item \url{/services/instagram/{NAME}}
    \item \url{/services/google/{NAME}}
    \item \url{/services/linkedin/{NAME}}
\end{itemize}