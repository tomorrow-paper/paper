\section{L'irruption des technologies}

\subsection*{Collecter : hier et aujourd'hui}

\paragraph{} Le terme de \emph{collecte} n'est pas récent ; en effet il nous viendrait vraisemblablement
du XIVe siècle et du terme \emph{collete} en latin, qui signifie la \guillemotleft levée des impositions\guillemotright,
soit la période pendant laquelle un collecteur était en fonction. Cette définition relève de plusieurs notions
qui nous intéressent plus particulièrement.

\paragraph{} Tout d'abord, la collecte c'est la levée d'un impôt. C'est donc que le collecté \emph{doit}
quelque chose au collecteur, pour une \emph{raison} particulière : mise à disposition de biens, services,
protection... La collecte est donc un échange entre deux entités. Cet échange est avant tout un contrat de
pouvoir : contre tel service, en échange de tel accès, de telle reconnaissance, le collecté accepte de
léguer une part de ses biens, de ce qu'il est ou ce qu'il a. Ce qui n'est pas le cas du collecteur qui est
dans une situation de \emph{prêt}, car il ne pourrait pas, en effet, mettre à disposition un service dont
il ne contrôle pas le flux. Cela l'empêcherait de pratiquer sa \emph{condition de collecteur}, n'étant pas en
\emph{situation de force} par rapport aux collectés.

\paragraph{} Il y a donc un rapport de pouvoir entre collecteur et collecté. Le collecteur s'apparente à un n\oe{}ud
parent, qui exerce le même contrôle sur tous ses noeuds enfants, qui eux-mêmes ne peuvent finalement qu'accepter
le pouvoir émis par ce dernier, à partir du moment où il n'existe pas d'autre collecteur offrant le \emph{même
service} pour une \emph{peine moindre}. On reconnait là une première forme de centralisation des données.
Au départ, la collecte concerne évidemment les revenus, l'argent d'un foyer ou d'un particulier. Car cet
argent est le pouvoir de vie (ou de \emph{survie}) d'un Homme, il doit accepter d'en laisser, d'en léguer un peu
s'il souhaite avoir accès à un bien. Mais l'argent étant assez mal réparti au sein des sociétés \cite{Richesses0},
tous ne peuvent se permettre de payer. Ce sont alors d'autres biens qui sont collectés : récoltes, créations
artisanales... C'est le produit de la \emph{technique} de l'Homme qui est récolté. Ce sont son \emph{temps} et
ses \emph{capacités techniques} qui lui permettent de payer.

\paragraph{} Néanmoins, ces collectes demandent une contribution matérielle de la part du collecté, et ce n'est pas
nécessairement le cas. Un exemple ancien serait celui des citoyens, en Grèce antique, qui devaient être répertoriés
dans des tables afin de témoigner de leur condition de citoyen. A titre exceptionnel, des femmes ou des esclaves
devenaient parfois citoyen, et la présence de leur nom sur une tablette leur permettait de l'affirmer et d'être
reconnu comme tel. Un parallèle peut être fait avec l'exemple plus récent des nationalités et des cartes d'identité :
avoir un passeport ou une carte d'identité, c'est être citoyen d'un pays. C'est donc bénéficier des \emph{lois} de ce
pays quelque soit le cadre physique (ou presque). Dans les deux cas, le citoyen ne paie pas son recensement par de l'argent,
des récoltes ou autres. En revanche, il est \emph{de son devoir} d'être recensé, sinon il ne pourrait bénéficier des 
conséquences de son recensement. On peut alors statuer qu'il y a des récoltes de données qui sont \emph{indirectement
obligatoires} pour l'Homme bien qu'elles ne soient pas \emph{matériellement coûteuses}.

\paragraph{} La collecte est donc, qu'elle soit directe ou indirecte, \emph{imposée}. Dans nos exemples précédents,
la collecte est toujours \emph{acceptée} et \emph{consciente}. Acceptée tout d'abord car il n'y a finalement pas le
choix : si je souhaite bénéficier de quelque chose, je dois pouvoir céder une contrepartie. C'est la nature même des
échanges humains - je ne suis prêt à te donner ceci qu'\emph{en échange de} cela. Il y a donc une interaction directe 
: deux humains intéragissent pour parvenir à un accord. Mais ce n'est justement que parce que cette collecte met en
jeu deux humains qu'elle peut être qualifiée de \emph{consciente} - parce que le collecté est capable de se dire à
un instant précis qu'\emph{il est effectivement en train de céder la contrepartie} à quelque chose. Ce n'est pas
toujours le cas et c'est pourquoi nous devons distinguer deux formes particulières d'acceptation de la collecte :
\emph{consciente} et \emph{inconsciente}.   

\subsection*{Collectes et consciences} \label{collect_data_conscious}

\paragraph{} Avant même de parler de conscience de la collecte, il est de bon ton de rappeler,
comme le précise C.G. Jung, qu'\guillemotleft Il est assez stérile d'étiqueter les gens et de les presser dans
des catégories\guillemotright. Par ici, on entend que l'\emph{expérience} même de la conscience est différente
pour tout un chacun. On parlera donc de "collecte affirmée" pour ce qui est conscient, et de "collecte non précisée"
pour ce qui est inconscient. Tout le monde n'est pas conscient que les publicités que l'on voit sur internet sont
ciblées ; et même si les utilisateurs y sont de plus en plus attentifs, il n'est pas indiqué à côté de chaque
publicité "Cette pub pour tel achat est présente car nous avons collecté vos dernières recherches et que plus d'un
quart concerne ce sujet ou des sujets proches". 

\paragraph{} Tout ce qui est inconscient est plus évident à aborder en premier lieu pour une raison simple : tout est
inconscient par définition. Prendre conscience de, c'est \emph{considérer} quelque chose, y porter \emph{son attention}.
Or, par nature, l'Homme ne porte attention qu'à une chose à la fois, et certainement pas à l'ensemble de son
environnement. Toute donnée doit donc, dans l'absolu, être collectée de façon inconsciente : de l'heure du réveil au
temps mis pour se préparer, l'itinéraire choisi pour aller au travail ou faire des courses, les lieux et les personnes
qui ont été fréquentées... L'ensemble des données générées au quotidien par tout être humain est collectable par
définition. Les questions qui se posent donc sont : comment, et quelles données sont pertinentes ?

\paragraph{} Afin d'apporter une réponse à la première question il est important de s'arrêter un instant sur le fait
suivant : lorsque l'on rend la collecte \emph{nécessaire pour le collecté} - comme dans l'exemple précédent sur la
citoyenneté - elle n'est plus \emph{inconsciente}, car elle est \emph{affirmée}. La nécessité de la collecte porte
l'attention du collecté sur cette dernière, qui devient alors par définition \emph{consciente}. Cela
implique donc qu'une collecte est \emph{inconsciente} uniquement si elle \emph{ne créé par de nécessité}
chez le collecté. On se rend alors immédiatement compte que c'est l'arrivée de nouvelles techniques et de la
technologie qui permet une collecte de données \emph{inconsciente}. En s'infiltrant dans les interstices de nos vies -
smartphone, ordinateur, internet, transports, paiements - la technologie \emph{traque tout} et \emph{tout le temps},
sans que nous ne nous en aperçevions. Une évolution importante de nos jours, et que nous détaillerons par la suite, est
celle de l'IoT (Internet of Things) et des objets connectés en général. Leur utilisation permet une collecte plus
poussée mais surtout \emph{plus proche} de l'utilisateur, et donc une récolte de données bien plus \emph{précise}. Or
nous verrons lorsque nous aborderons le thème du Machine Learning qu'il existe un lien fort entre la précision d'une
donnée et la qualité du service qui pourra être fourni à un utilisateur (voir \ref{select_data_ml}).

\paragraph{} Puisque ces données sont partout, et qu'elles vont être de plus en plus récupérées, il est devenu
une question importante que celle du \emph{stockage} de ces données. Afin d'optimiser ce stockage, et avant même de
penser aux techniques physiques utilisées pour garder des données, il faut effectuer un tri dans ce qui sera récupéré.
Pourtant, ce tri est aujourd'hui davantage effectué une fois que la donnée est récupérée. En effet, afin de déterminer
si une donnée est pertinente, il suffit de se demander si elle apporte des informations suffisantes à la récupération
d'un bénéfice futur. C'est le principe même de \emph{collecte} : le collecté consomme et le collecteur en tire \emph{
une valeur}, directe ou indirecte. C'est donc bien qu'il y a une question de \emph{monétisation} de la donnée. La
problématique n'est plus donc de savoir quoi récupérer mais plutôt de comment exploiter ce qui à été récupéré.

\paragraph{} Dès lors que nous parlons de création de valeur, cela replace le contexte de la collecte dans un échange
\emph{conscient}, ou plutôt \emph{affirmé}, de la donnée. Il faut au départ que le collecteur produise un \emph{service},
qu'il existe un \emph{contrat} entre collecteur et collecté pour que la création de données ait lieu. Une matérialisation
simple et actuelle de cela se retrouve lorsqu'un utilisateur installe une application sur son smartphone - il est
alors invité à accepter ou non que l'application ait à son tour accès à un certain nombre de choses (medias, gallerie,
données personnelles, agenda...) sur le smartphone. C'est la partie \emph{consciente} du choix. Une fois qu'elle est
\emph{acceptée}, la partie \emph{inconsciente est alors acceptée également par défaut}. En effet le collecté accepte que
l'utilisation qu'il va faire de l'application soit collectée \emph{de facto}, par la \emph{nécessité d'accès à un service}.

\subsection*{Prototype}
\label{prototype_recuperator}

\paragraph{} Pour étayer notre propos, un prototype servira de fil rouge tout au long de votre lecture : \lstinline{Tomorrow}
(\url{https://github.com/tomorrow-paper}). Nous détaillerons à la fois les objectifs remplis par chacun de ses modules
ainsi que les choix techniques que nous avons effectués.

\paragraph{} Nous avons choisi le \lstinline{Rust} \cite{Rust0} comme principal langage de programmation, car il allie une syntaxe
de haut niveau moderne avec tous les avantages en terme de performance apportés par des exécutables natifs compilés.
Multi-paradigmes, l'une des grandes forces de ce langage est sa gestion extrêmement stricte de la mémoire, introduisant
des concepts nouveaux comme l'\emph{ownership} (propriété) et le \emph{borrowing} (emprunt) d'espaces et de références 
mémoires permettant d'assurer que le code produit est exempt de fuites, \emph{data races}, problèmes de parallélisation...
Son gestionnaire de dépendances, \lstinline{cargo}, ainsi que son riche écosystème de \emph{crates} (bibliothèques) sont
à même de réconcilier les plus réfractaires avec les langages de programmation système.

\paragraph{} Pour accompagner ce premier chapitre, nous avons souhaité mettre en place un service permettant d'aggréger
des données depuis n'importe quelle source pour ensuite pouvoir les consommer de manière normalisée.
Nous nous intéresserons ici à la récupération de données personnelles, et avons donc pour cela restreint notre champs
d'action à quelques services pertinents :

\begin{itemize}
    \item Google : récupération des suggestions de pages liées à la personne concernée par le
    premier moteur de recherche mondial (\url{https://encrypted.google.com/search?q={SEARCH_TERMS}}) ;
    \item Google Maps API : récupération des coordonnées géographiques associées à une adresse
    (\url{https://maps.google.com/maps/api/geocode/json?address={ADDRESS}}) ;
    \item Facebook : recherche et récupération des données publiques des profils utilisateurs
    (\url{https://www.facebook.com/public/{PEOPLE}}).
\end{itemize}

\paragraph{} De manière à préserver la séparation des responsabilités des composants de notre prototype, plusieurs modules
ont été développés : 

\begin{itemize}
    \item \lstinline{tomorrow-core} (\url{https://github.com/tomorrow-paper/tomorrow-core}) : contient la logique et les
    structures de données communes à tous les composants du prototype (gestion d'erreur, type résultat, ...) ;
    \item \lstinline{tomorrow-http} (\url{https://github.com/tomorrow-paper/tomorrow-http}) : contient les interfaces et
    implémentations de la logique de requêtage à travers le protocole HTTP.
    \item \lstinline{tomorrow-recuperator} (\url{https://github.com/tomorrow-paper/tomorrow-recuperator}) : contient les interfaces
    de récupération de données.
\end{itemize}

\paragraph{} Ce découpage n'est pas anodin. Il nous permet par exemple de développer, pour chaque service cible, un module
dédié implémentant les interfaces définies dans \lstinline{tomorrow-recuperator}  :

\begin{lstlisting}
use tomorrow_core::Result;

pub trait Request {}
pub trait Response {}

pub trait Recuperator<Req, Res> where Req: Request, Res: Response {
    fn compute(&self, request: Req) -> Result<Res>;
}\end{lstlisting}

\paragraph{}\label{tomorrow-api} Ces différents \emph{récupérateurs} sont ensuite utilisés pour constituer un composant à plus haut niveau
d'abstraction : \lstinline{tomorrow-api} (\url{https://github.com/tomorrow-paper/tomorrow-api}). API REST produisant des
documents JSON, elle sert de façade pour d'éventuelles applications ou un usage à des fins de recherche par un être humain. 
Pour cette première version de notre prototype, aucun \emph{état} ni aucune donnée ne sont persistés. Les informations 
sont lues à un instant \emph{T} quand l'utilisateur en fait la demande et lui sont immédiatement renvoyées par l'API.
Les \emph{endpoints} actuels sont les suivants (tous les modules peuvent être trouvés sur notre dépôt) :

\begin{itemize}
    \item \url{/services/maps/{ADDRESS}} : Google Maps grâce au module \lstinline{recuperator-google-maps} ;
    \item \url{/services/search/{QUERY}} : Google Search grâce au module \lstinline{recuperator-google} ;
    \item \url{/services/facebook/public/{PEOPLE}} : Facebook Public Search grâce au module \lstinline{recuperator-facebook}.
\end{itemize}

\paragraph{} Nos objectifs en développant ce service étaient triples. Tout d'abord, il se devait d'être agnostique en terme
d'utilisation : une API REST ne repose que sur le protocole HTTP, et le format JSON est aujourd'hui un standard en terme
d'encodage d'information. Ensuite, nous souhaitions l'architecturer de manière évolutive : pour récupérer les données
depuis un nouveau service, il suffit de créer un module implémentant les interfaces \lstinline{tomorrow-recuperator} et
de l'intégrer à l'API derrière un nouvel \emph{endpoint}. Enfin nous espérons qu'il vous aidera à prendre conscience que
rien n'empêche l'exploitation d'une donnée accessible publiquement sur Internet de nos jours. Car peu importe le nombre
de services différents que nous pourrions intégrer à ce prototype : c'est sa nature même qui doit vous interpeler.