\section{Un Réseau}
\paragraph{Références} \cite{DarkWeb0}

\paragraph{Réseau \emph{nominal}} "Idéal" de la mise en place d'un Réseau universel, omniprésent
et omniscient. Un tel système global impliquerait la mise en place d'un système hautement
distribué, voire \emph{"participatif"}.

\paragraph{Réseaux parallèles} Verra-t-on un jour la disparition des réseaux parallèles
(TOR, blockchains, ...) ? Ces technologies \emph{disruptives} seront-elles un jour vouées
à disparaitre ou au contraire à se développer avec d'autant plus d'ardeur ? Peut-on alors
parler d'une neutralisation du Réseau par le réseau ?

\paragraph{Objectif} L'objectif ici est de mettre en place un prototype qui se chargera de la récolte des données
de manière insidieuse et de leur mise ne réseau. Un logiciel, installé sur un ou plusieurs pc,
pourra envoyer des données à une entité centrale ou bien échanger entre eux.
