\section{Des contraintes architecturales}

\paragraph{} Cependant, le choix de l'architecture que l'on souhaiterait mettre en place de manière \emph{globale} doit 
être sagement réfléchi. En effet, les contraintes d'un tel déploiement et de l'ampleur de l'utilisation de ce système
doivent être clairement identifiées et adressées dans le choix même de l'architecture.

\paragraph{} Le premier obstacle à une mise en place d'un service à grande échelle est la \emph{scalabilité}. \cite{Scalability0}

\paragraph{TODO} Résilience.

\paragraph{TODO} Dans un système fortement distribué, la latence est le prix à payer de la résilience. Elle nécessite une 
disponibilité du service dans de nombreuses zones géographiques, réduisant par là même les risques de sinistre. Il est alors
nécessaire, soit de mettre en place un cache applicatif ou un CDN (\emph{Content Delivery Network}) pour optimiser les 
accès aux ressources ; des serveurs proxy ou des tunnels dédiés pour accéder directement au service ; déployer plusieurs
instances de l'applciation dans les différentes régions couvertes, ce qui nécessite une gestion adéquat des \emph{états}
au sein de l'application ainsi qu'un moyen de resynchroniser l'ensemble des données de manière à les aggréger.

\paragraph{} La réponse apportée à une contrainte d'architecture peut elle-même lever d'autres questions \& challenges.

\paragraph{} Pour étudier les solutions que l'on peut apporter à ces différentes contraintes, nous vous proposons d'étudier
ci-dessous trois architectures qui adressent chacune d'entre elles de manière différente, disposant de leurs propres
avantages et inconvénients.


\paragraph{Architecture Monolithique}

\cite{Microservices3} \cite{Microservices4} \cite{Microservices5} \cite{Microservices6}

\paragraph{TODO} Étude de cas : les monolithes. Développement et déploiement simple. Maintenance et évolution dépendant fortement
du couplage et de la qualité générale de la codebase. Scalabilité complexe, d'autant plus si le monolithe conserve un 
état. Association forte à une stack technologique qu'il devient laborieux de faire évoluer. Faible latence au sein de 
l'application. Nécéssité de mettre en place une architecture supplémentaire (2\up{nd} layer) pour prendre en charge et
répartir la charge.


\paragraph{Microservices}

\cite{Microservices0} \cite{Microservices1} \cite{Microservices2}

\paragraph{TODO} Étude de cas : les microservices. Plus grande scalabilité et réutilisabilité si le couplage entre les composants
est faible ; de même celà permet une forte résilience du système, qui peut rester - partiellement - opérationel quand
bien même l'un des composants viendrait à disparaître. Induit généralement une augmentation de la latence applicative.
Gestion complexe des versions des différents services et augmentation de la complexité du déploiement. Comment, à l'échelle
mondiale, serait gérée une telle infrastructure ?


\paragraph{Blockchain}

\cite{Blockchain0} \cite{Blockchain1}

\paragraph{TODO} Étude de cas : la blockchain. Distribution des acteurs et des données, grande résilience du réseau mais scalabilité
du système et latence - en fonction de la technologie - potentiellement accompagnée d'un coût d'usage élevé.

\paragraph{TODO} Problèmes de scalabilité à prendre en compte avec les technologies blockchain :

\begin{itemize}
    \item ETH : CryptoKitties (\url{https://www.cryptokitties.co/}) qui a embourbé le réseau en novembre 2017. 
    Développement d'une solution de scaling : le \emph{Sharding}, \url{https://github.com/ethereum/sharding/blob/develop/docs/doc.md}
    \item BTC : augmentation des frais de transaction due à l'intense usage du réseau.
    Soft fork avec SegWit, servant de bases à deux solutions au problème de scalabilité : SegWit2x, hard fork du réseau,
    et Lightning Network, une solution off-chain permettant l'ouverture et la fermeture de canaux transactionnels par dessus
    le réseau Bitcoin.
\end{itemize}


\paragraph{Un Réseau hétérogène}

\paragraph{TODO} L'hétérogénité n'est pas un frein à l'unité. L'unité c'est le tout.
