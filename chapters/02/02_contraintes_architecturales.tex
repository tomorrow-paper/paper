\section{Des contraintes architecturales}

\paragraph{Brouillon}

\begin{enumerate}
    \item Quelles contraintes architecturales pour la mise en place d'un réseau universel ?
    Nécessité d'adresser des problèmes de latence, scalabilité et résilience.
    \item Étude de cas : les microservices. Plus grande scalabilité et réutilisabilité si le couplage entre les composants
    est faible ; de même celà permet une forte résilience du système, qui peut rester - partiellement - opérationel quand
    bien même l'un des composants viendrait à disparaître. Induit généralement une augmentation de la latence applicative.
    Gestion complexe des versions des différents services et augmentation de la complexité du déploiement. Comment, à l'échelle
    mondiale, serait gérée une telle infrastructure ?
    \item Étude de cas : la blockchain. Distribution des acteurs et des données, grande résilience du réseau mais scalabilité
    du système et latence - en fonction de la technologie - potentiellement accompagnée d'un coût d'usage élevé.
\end{enumerate}

\paragraph{} Dans un système fortement distribué, la latence est le prix à payer de la résilience.

\paragraph{} Problèmes de scalabilité à prendre en compte avec les technologies blockchain :

\begin{itemize}
    \item ETH : CryptoKitties (\url{https://www.cryptokitties.co/}) qui a embourbé le réseau en novembre 2017. 
    Développement d'une solution de scaling : le \emph{Sharding}, \url{https://github.com/ethereum/sharding/blob/develop/docs/doc.md}
    \item BTC : augmentation des frais de transaction due à l'intense usage du réseau.
    Soft fork avec SegWit, servant de bases à deux solutions au problème de scalabilité : SegWit2x, hard fork du réseau,
    et Lightning Network, une solution off-chain permettant l'ouverture et la fermeture de canaux transactionnels par dessus
    le réseau Bitcoin.
\end{itemize}

\paragraph{Vecteurs d'adhésion} OS, Suprématie Windows/OSX