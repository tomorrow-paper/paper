\chapter{Un Système pour les gouverner tous}
\paragraph{Objet} Systèmes possibles ; Réseau Distribué
\paragraph{Technique} Un système distribué : scalabilité et résilience applicative
\paragraph{Références}
\cite{Deleuze0}
\cite{Foucault0}
\cite{Negri0}
\cite{Pieces0}
\cite{ProgrammableCity0}
\cite{ProgrammableCity1}
\cite{PsychoPass}

\section{Un Réseau}
\paragraph{Références} \cite{DarkWeb0}

\paragraph{Réseau \emph{nominal}} "Idéal" de la mise en place d'un Réseau universel, omniprésent
et omniscient. Un tel système global impliquerait la mise en place d'un système hautement
distribué, voire \emph{"participatif"}.

\paragraph{Réseaux parallèles} Verra-t-on un jour la disparition des réseaux parallèles
(TOR, blockchains, ...) ? Ces technologies \emph{disruptives} seront-elles un jour vouées
à disparaitre ou au contraire à se développer avec d'autant plus d'ardeur ? Peut-on alors
parler d'une neutralisation du Réseau par le réseau ?

\paragraph{Objectif} L'objectif ici est de mettre en place un prototype qui se chargera de la récolte des données
de manière insidieuse et de leur mise ne réseau. Un logiciel, installé sur un ou plusieurs pc,
pourra envoyer des données à une entité centrale ou bien échanger entre eux.


\section{Des contraintes architecturales}
\paragraph{Des contraintes à prendre en compte} Scalabilité, résilience..
\paragraph{Vecteurs d'adhésion} OS, Suprématie Windows/OSX

\paragraph{} Do it Yourself : Mettez en place votre propre télémétrie ! (Keylogger ?)


\section{L'Homme gouverne l'Homme}
\paragraph{Références} \cite{GhostInTheShell}

\paragraph{Un rêve devenu réalité ?} Systèmes distribués, mise en oeuvre à grande échelle.
À travers le Réseau, c'est l'Homme qui devient omniscient.

\paragraph{L'Homme} Chaque foyer comme noeud du réseau, chaque individu comme neurone du
système. La surveillance n'est pas \emph{automatique}, le système ne s'auto-alimente pas :
c'est à travers \emph{nos} interactions qu'il se nourrit et prend de l'ampleur (ajout de 
contenus, d'informations de plus en plus exhaustive sur nous et nos connaissances, ...).

\paragraph{Objectif} Attention aux redites - le but ici est de montrer que l'Homme est à la fois
consommateur et producteur des informations transitant sur le réseau.