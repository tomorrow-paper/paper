\section{L'Homme gouverne l'Homme}

\paragraph{} Les hommes possèdent donc, aujourd'hui, \emph{toutes} les architectures et réseaux
nécessaires afin d'être mis en relation. L'échange de données, l'accès à des plateformes
\emph{centralisées}, sont \emph{instantanés}. Les notions de consommation et de production
sont maintenant intrinséquement liées : on ne peut plus \emph{consommer sans produire}, ni même
\emph{produire sans consommer}. L'Homme est devenu \emph{consommacteur} de technologies.

\paragraph{} A travers la mise en \oe{}uvre à grande échelle, l'Homme peut être
\emph{partout, tout le temps}. Il en découle un statut particulier de l'Homme 
technologique : producteur et consommateur \emph{intemporel}, l'Homme devient \emph{Omniscient}.
Il fait partie \emph{du Réseau}, parce qu'il est présent sur \emph{les réseaux}. Notre identité
est avant tout constituée des \emph{multiples briques d'identité} que nous acceptons de
\emph{partager}.

\paragraph{} Mais faire partie du réseau, n'est-ce pas déjà en soi \emph{y participer} ? L'\oe{}uvre
de Masamune Shirow, \emph{Ghost In the Shell} \cite{GhostInTheShell}, nous fournit l'enseignement
suivant : dans la société sur-technologisée, ce n'est plus \emph{l'action} qui détermine la
\emph{participation} - c'est \emph{l'existence}. Dès lors, si l'Homme souhaite se \emph{libérer}
des contraintes technologiques, il ne lui reste qu'une possibilité : \emph{ne pas exister} au
sein des réseaux.

\paragraph{} Que cela implique-t-il ? Selon nous, une \emph{autarcie complète}. Nous avons vu que la
collecte est maintenant \emph{insidieuse} : au-delà des réseaux, \emph{tout est collecté} (transports,
santé, citoyenneté) et ne permet même plus à l'Homme d'agir \emph{librement}, en respectant son
\emph{anonymat}. Pourtant même cette autarcie nous semble \emph{difficilement accessible}. Chaque foyer,
chaque personne, chaque entité forme un n\oe{}ud du réseau. La surveillance n'est pas automatique : 
elle se nourrit de nos interactions. C'est donc bien que nous devons \emph{cesser d'intéragir} si nous
souhaitons échapper à la part de \emph{règles et de contrôle} que porte chacun : comme le précisait Sartre,
finalement, \guillemotleft L'enfer, c'est les autres\guillemotright. \cite{Sartre0}

\paragraph{} -----------------------------------

\paragraph{} Mais la surveillance naît de la crainte et des peurs de l'Homme. 
Il souhaite que les règles qu'il se fixe s'appliquent à tous. Cela se constate
dans les réactions des gouvernements, souvent dépassés par les avancées technologiques.

\paragraph{} Quels enseignements à tirer des gouvernements qui essaient, eux, 
d'adopter la technologie pour mieux la réguler ? Exemple de l'e-nationalité
dans la blockchain. (\url{https://e-resident.gov.ee/})

\paragraph{} -----------------------------------

\paragraph{Objectif} Attention aux redites - le but ici est de montrer que l'Homme est à la fois
consommateur et producteur des informations transitant sur le réseau.

\paragraph{} -----------------------------------

\paragraph{} Mais l'Homme s'est toujours gouverné lui-même. En société ou seul, il se fixe
\emph{des règles}, qui lui permettent d'évoluer, de se fixer un objectif, \emph{d'avancer}.
Ce comportement est le propre de l'Homme, et est une conséquence directe de son intelligence.
En choisissant de la déléguer \emph{aux machines}, il choisit de déléguer aussi
bien une partie du contrôle qu'il exerce \emph{sur lui} que de celui qu'il exerce \emph{sur autrui}.