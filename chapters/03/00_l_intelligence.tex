\chapter{L'intelligence envahissante de votre quotidien}
\paragraph{Objet} Développement technique
\paragraph{Technique} Intelligence, traitement des données
\begin{itemize}
    \item Tendances actuelles
    \item prévisions
    \item Analyses psychologiques
    \item Géolocalisation
\end{itemize}

\paragraph{Références}
\cite{Asimov:0}
\cite{Moore:0}
\cite{MachineLearning:0}
\cite{MachineLearning:1}
\cite{ProgrammableCity:0}
\cite{ProgrammableCity:1}
\cite{GhostInTheShell}
\cite{PsychoPass}

\section{Intelligence as a Service}
\paragraph{Références} 

\paragraph{Qu'est-ce que l'IA ? Quelle est sa place ?} Nous entendons parler
de plus en plus d'intelligence artificielle, mais quels sont ses principes
fondateurs ? Comment ce domaine s'est-il démocratisé ? Quels exemples avons-nous
de son utilisation dans des systèmes actuels ? 

\paragraph{Quelles contraintes ?} Quelles sont les conditions techniques nous
permettant d'intégrer de l'IA dans des systèmes ? Y a-t-il des contraintes de
volumétrie, de scalabilité, de puissance de calculs ? Nous verrons comment
l'intégration de systèmes intelligents s'est faire avec plus ou moins de succès
selon et les projets, ainsi que les solutions et évolutions qui ont été apportéées,
permettant de simplifier la mise en place et l'utilisation d'IAs. 

\paragraph{Quels apports et quelle utilité ?} Quels sont les apports \emph{concrets}
de l'IA ? Quels domaines sont les plus concernés ? Est-il possible de distinguer
des \emph{patterns d'implémentation} permettant d'apporter des réponses systématiques
à des problèmes connus ?

%%

\section{Machine Learning}
\paragraph{Références} \cite{AlphaGo:0} \cite{AlphaGo:1}

\paragraph{Différences et similitudes avec IA} En quoi consiste le ML ? Bien qu'il soit
directement lié à l'IA, quelles sont ses différences et ses similitudes avec cette dernière ?

\paragraph{Evolution logique et inévitable de l'IA} 

\paragraph{Systèmes intelligents} Lorem ipsum

%%

\section{Usages, dérives et éthique}
\paragraph{Références} \cite{Asimov:0} \cite{Damasio:0}

\paragraph{Déterminisme, cas d'utilisations} Lorem ipsum

\paragraph{Eugénisme, dérives} Lorem ipsum

\paragraph{Améliorations de vie et sociales} Lorem ipsum