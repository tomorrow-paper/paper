\section{Intelligence as a Service}
\paragraph{Références} 

\paragraph{Qu'est-ce que l'IA ? Quelle est sa place ?} Nous entendons parler
de plus en plus d'intelligence artificielle, mais quels sont ses principes
fondateurs ? Comment ce domaine s'est-il démocratisé ? Quels exemples avons-nous
de son utilisation dans des systèmes actuels ?

\paragraph{} Une définition formelle de l'IA comme part des mathématiques.
Les débuts de l'IA, explication d'algorithmes simples.

\paragraph{} Quel avenir pour l'IA ? Un peu de science-fiction. Ce qui arrive et ce qui 
n'arrive pas. La révolution du smartphone. La démocratisation de l'IA dans le smartphone.

\paragraph{Quelles contraintes ?} Quelles sont les conditions techniques nous
permettant d'intégrer de l'IA dans des systèmes ? Y a-t-il des contraintes de
volumétrie, de scalabilité, de puissance de calculs ? Nous verrons comment
l'intégration de systèmes intelligents s'est faire avec plus ou moins de succès
selon et les projets, ainsi que les solutions et évolutions qui ont été apportéées,
permettant de simplifier la mise en place et l'utilisation d'IAs.

\paragraph{} Systèmes simples et systèmes complexes. Des patterns gourmands et
des optimisations parfois compliquées. Parallèle avec le bruteforce. Que donnerait 
une tentative d'application de l'IA (pas du ML !) au calcul du BTC ?

\paragraph{Quels apports et quelle utilité ?} Quels sont les apports \emph{concrets}
de l'IA ? Quels domaines sont les plus concernés ? Est-il possible de distinguer
des \emph{patterns d'implémentation} permettant d'apporter des réponses systématiques
à des problèmes connus ?

\paragraph{} L'utilisation de l'IA dans différents domaines: jeu vidéo, informatique,
robotique, médecine, finance, domotique (pathfinding aspis)...

\paragraph{} Le principe de l'heuristique. La recherche de la solution optimale.
L'exemple d'alphaGoZero comme apprentissage à partir du néant (à developper en ML).
La possibilité de l'application à tout domaine.

\paragraph{Objectif} L'objectif est d'ajouter ici des traitements de données automatisés au section
du Réseau.