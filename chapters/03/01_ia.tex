\section{Intelligence as a Service}
\paragraph{Références} \cite{AI0} \cite{AI1}

\paragraph{Quelle est sa place ?} Comment ce domaine s'est-il démocratisé ? Quels exemples avons-nous
de son utilisation dans des systèmes actuels ?

\paragraph{Qu'est-ce que l'IA ?}

\paragraph{} Il est important en premier lieu de comprendre comment est né le terme d'intelligence
artificielle et ce qu'il implique techniquement. L'IA est un domaine de l'informatique et se définit
comme l'étude des \emph{agents intelligents} : c'est-à-dire tout appareil en mesure de perçevoir son
environnement, et de prendre les actions en conséquence afin de maximiser ses chances de succès pour
le même objectif donné. \cite{AI0} Elle s'applique également aux fonctions cognitives qui
imitent l'humain, à l'exemple de \emph{l'apprentissage} ou de la \emph{résolution de problèmes}.

\paragraph{} Néanmoins, l'intelligence artificielle n'a pas toujours eu le même sens, ni comporté
tous les domaines présents aujourd'hui. Pour certains, elle désignerait uniquement ce qui n'a
pas encore été réalisé, comme la création d'une machine dotée d'une intelligence humaine et autonome.
On peut comprendre en ce sens que l'IA s'attache à essayer de conçevoir et de comprendre des faits
dans des domaines vastes et en perpétuelle évolution. Nous verrons par la suite que, même si
l'utilisation d'intelligences s'est \emph{démultipliée} de manière \emph{exponentielle} ces vingt
à soixante dernières années, le terme est encore utilisé parfois à tort ou de manière imprécise 
pour désigner des concepts mal compris. Cela s'explique notamment par l'utilisation abusive du terme 
par les médias, qui en font \emph{un buzzword}, et \emph{désinforment parfois} l'auditeur.
Par ailleurs, nous avons vu que la littérature et la science-fiction en particulier envisagent des
scénarii qui ne se révèlent pas nécessairement exacts. 

\paragraph{Objet d'étude} Les domaines d'études usuels de la recherche en intelligence artificielle sont le
\emph{raisonnement}, la \emph{perception}, l'\emph{apprentissage}, la \emph{connaissance}, la
\emph{prévision}, le \emph{traitement du langage naturel} (ou TLN, Natural Language Processing en anglais),
et en robotique l'abileté de \emph{se déplacer}, ou encore de \emph{déplacer} ou \emph{manipuler} des
objets. Son application nécessite l'utilisation d'outils comme l'\emph{optimisation}, la \emph{théorie des graphes},
mais aussi des méthodes statistiques et des notions de probabilité. 

\paragraph{} ------------------------------------

\paragraph{} Le test de Turing, la recherche d'adaptabilité, les approches symbolique et connexionniste

\paragraph{} ------------------------------------

\paragraph{} L'utilisation de l'IA dans différents domaines: jeu vidéo, informatique,
robotique, médecine, finance, domotique...

\paragraph{} ------------------------------------

\paragraph{} Les métaheuristiques d'optimisation, optimums locaux et globaux, exemples. 
Leur domaine d'application

\paragraph{} ------------------------------------

\paragraph{} La recherche de chemins. Théorie des graphes, Bellman-Ford, Dijkstra, A*.
Ouverture sur le GPS et les services type Uber

\paragraph{} ------------------------------------

\paragraph{} La révolution du smartphone. La démocratisation de l'IA dans le smartphone.
Quel avenir pour l'IA ?

\paragraph{} ------------------------------------

\paragraph{Quelles contraintes ?} Quelles sont les conditions techniques nous
permettant d'intégrer de l'IA dans des systèmes ? Y a-t-il des contraintes de
volumétrie, de scalabilité, de puissance de calculs ? Nous verrons comment
l'intégration de systèmes intelligents s'est faire avec plus ou moins de succès
selon et les projets, ainsi que les solutions et évolutions qui ont été apportéées,
permettant de simplifier la mise en place et l'utilisation d'IAs.

\paragraph{} Systèmes simples et systèmes complexes. Des patterns gourmands et
des optimisations parfois compliquées. Parallèle avec le bruteforce. Que donnerait 
une tentative d'application de l'IA (pas du ML !) au calcul du BTC ?

\paragraph{Quels apports et quelle utilité ?} Quels sont les apports \emph{concrets}
de l'IA ? Quels domaines sont les plus concernés ? Est-il possible de distinguer
des \emph{patterns d'implémentation} permettant d'apporter des réponses systématiques
à des problèmes connus ?

\paragraph{Objectif} L'objectif est d'ajouter ici des traitements de données automatisés au section
du Réseau.

