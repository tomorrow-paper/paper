\chapter{Conclusion}

\paragraph{} C'est donc chaque jour que nous délaissons notre puissance au profit des
technologies. À travers Internet, les smartphones, les objets connectés, nous acceptons
à chaque instant de \emph{léguer} et de \emph{déléguer} une part de nous-même à la 
technologie. 

\paragraph{} Allégorie du contrôle, propre de l'Homme, les données que ces
interactions génèrent sont \emph{traquées partout et tout le temps}, et les quantités
gigantesques que cela représente dépassent maintenant l'entendement. Ces données sont
générées par l'Homme \emph{à chaque instant}, sans plus \emph{aucune interruption}. 

\paragraph{} L'individu se conceptualise en termes technologiques. Il est devenu un
individu technique, \emph{un profil}. L'image que l'on cherche à donner à travers la
technologie ne reflète que celle que l'on souhaite adopter. Et au-delà du \emph{Moi}
qui est devenu le centre des intérêts, l'Homme reste à la fois \emph{connecté} mais
\emph{paradoxalement seul}.

\paragraph{} Seul, il constitue un des noeuds du réseau \emph{omniprésent} que représente
déjà Internet. Et il en est \emph{l'acteur principal} : c'est sa projection à travers
les services \emph{qu'il crée et qu'il utilise} qui va former son \emph{Moi technologique}.

\paragraph{} Et ces services forment eux-mêmes des réseaux par leurs interconnexions : la facilité
d'accès est avant tout centrée sur \emph{l'utilisateur}. Mais cette facilité n'est qu'une interface
\emph{créée par le facilitateur} pour servir ses propres intérêts. Ainsi les services se doivent
d'évoluer vers un \emph{écosystème commun} s'ils souhaitent asseoir leur \emph{universalité},
non au détriment de \emph{l'individu}. 

\paragraph{} Mais l'Homme s'est toujours gouverné lui-même. En société ou seul, il se fixe
des \emph{règles}, qui lui permettent d'évoluer, de se fixer un objectif, \emph{d'avancer}.
Ce comportement est le propre de l'Homme, et est une conséquence directe de son intelligence.
En choisissant de la déléguer \emph{aux machines}, il choisit de déléguer aussi
bien une partie du contrôle qu'il exerce sur \emph{lui} que de celui qu'il exerce sur \emph{autrui}.

\paragraph{} C'est donc aux \emph{systèmes} que le contrôle est délégué, et nous qualifions
nous-même ces systèmes \emph{d'intelligents}. Embarquée partout : smartphones, transports,
médical, finance, robotique ; l'Intelligence Artificielle est un formidable vecteur
de \emph{découvertes}. La puissance des IA à longtemps été fantasmée, et pourtant 
certaines \emph{dépassent} maintenant l'Homme dans des problèmes d'une complexité \emph{infinie}. 

\paragraph{} Le Machine Learning en est l'exemple le plus marquant des dix dernières
années. En reproduisant le fonctionnement du \emph{cerveau humain}, les machines sont
arrivées à faire ce qui était alors impossible : \emph{apprendre}. Et leur apprentissage
dépasse les attentes humaines, aussi bien dans la vitesse de progression que dans la
\emph{réelle application technique}. Cette avancée technologique se cristallise et 
atteint son paroxysme à travers les dernières découvertes du Machine Learning : la
machine apprend \emph{mieux, sans l'aide de l'Homme}. 

\paragraph{} Toutes les dérives négatives, scenarii catastrophe issus de notre imagination,
sont déjà \emph{à portée de main}. Les données sont \emph{accessibles}, les réseaux \emph{universels},
les machines \emph{humanisées}. Il ne tient qu'à une société, un état, un
individu, de choisir la \emph{manière} dont il souhaite \emph{utiliser et être
utilisé} par ces outils.

\paragraph{} La balle est donc dans \emph{notre camp}, nous êtres humains. Quel sera
notre choix pour la société de demain ? Il ne tient qu'à nous de saisir
l'opportunité de l'élan technologique pour \emph{créer, aider}, et inventer
intelligemment notre espace virtuel et réel de demain. Mais la dissociation entre
les deux est encore mal faite, car nous ne sommes que les \emph{premières générations}
à être confrontées à de tels enjeux. Il s'agit maintenant de trouver des techniques
\emph{non-intrusives}, permettant de concilier \emph{innovation bénéfique} et
respect de \emph{l'individualité}.

\paragraph{} Cependant rien ne nous dit qu'il n'existera pas, demain, \emph{un Système pour nous
gouverner tous}.