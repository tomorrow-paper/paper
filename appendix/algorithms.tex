\newpage
\section{Algorithmes}
\label{algorithms}

\subsection{Recherche en profondeur} 
\paragraph{Référence} \cite{Algorithm2}
\begin{lstlisting}
using System;
using System.Collections.Generic;

namespace KoderDojo.Examples {
    public class Graph<T> {
        public Graph() {}

        public Graph(
            IEnumerable<T> vertices,
            IEnumerable<Tuple<T,T>> edges
        ) {
            foreach(var vertex in vertices)
                AddVertex(vertex);

            foreach(var edge in edges)
                AddEdge(edge);
        }

        public Dictionary<T, HashSet<T>> AdjacencyList { get; } =
            new Dictionary<T, HashSet<T>>();

        public void AddVertex(T vertex) {
            AdjacencyList[vertex] = new HashSet<T>();
        }

        public void AddEdge(Tuple<T,T> edge) {
            if (AdjacencyList.ContainsKey(edge.Item1) &&
                AdjacencyList.ContainsKey(edge.Item2)) {
                AdjacencyList[edge.Item1].Add(edge.Item2);
                AdjacencyList[edge.Item2].Add(edge.Item1);
            }
        }
    }

    public class Algorithms {
        public HashSet<T> DFS<T>(Graph<T> graph, T start) {
            var visited = new HashSet<T>();

            if (!graph.AdjacencyList.ContainsKey(start))
                return visited;
                
            var stack = new Stack<T>();
            stack.Push(start);

            while (stack.Count > 0) {
                var vertex = stack.Pop();

                if (visited.Contains(vertex))
                    continue;

                visited.Add(vertex);

                foreach(var neighbor in graph.AdjacencyList[vertex])
                    if (!visited.Contains(neighbor))
                        stack.Push(neighbor);
            }

            return visited;
        }
    }
}
\end{lstlisting}    

\newpage
\subsection{Recherche en largeur}
\paragraph{Référence} \cite{Algorithm1}
\begin{lstlisting}
using System.Collections.Generic;

namespace KoderDojo.Examples {
    public class Algorithms {
        public HashSet<T> BFS<T>(Graph<T> graph, T start) {
            var visited = new HashSet<T>();

            if (!graph.AdjacencyList.ContainsKey(start))
                return visited;
                
            var queue = new Queue<T>();
            queue.Enqueue(start);

            while (queue.Count > 0) {
                var vertex = queue.Dequeue();

                if (visited.Contains(vertex))
                    continue;

                visited.Add(vertex);

                foreach(var neighbor in graph.AdjacencyList[vertex])
                    if (!visited.Contains(neighbor))
                        queue.Enqueue(neighbor);
            }

            return visited;
        }
    }
}
\end{lstlisting}

\newpage
\subsection{Bellman-Ford}
\paragraph{Référence} \cite{Algorithm4}
\begin{lstlisting}
# Python program for Bellman-Ford's single source 
# shortest path algorithm.
 
from collections import defaultdict
 
#Class to represent a graph
class Graph:
 
    def __init__(self,vertices):
        self.V= vertices #No. of vertices
        self.graph = [] # default dictionary to store graph
  
    # function to add an edge to graph
    def addEdge(self,u,v,w):
        self.graph.append([u, v, w])
         
    # utility function used to print the solution
    def printArr(self, dist):
        print("Vertex   Distance from Source")
        for i in range(self.V):
            print("%d \t\t %d" % (i, dist[i]))
     
    # The main function that finds shortest distances from src to
    # all other vertices using Bellman-Ford algorithm.  The function
    # also detects negative weight cycle
    def BellmanFord(self, src):
 
        # Step 1: Initialize distances from src to all other vertices
        # as INFINITE
        dist = [float("Inf")] * self.V
        dist[src] = 0
 
 
        # Step 2: Relax all edges |V| - 1 times. A simple shortest 
        # path from src to any other vertex can have at-most |V| - 1 
        # edges
        for i in range(self.V - 1):
            # Update dist value and parent index of the adjacent vertices of
            # the picked vertex. Consider only those vertices which are still in
            # queue
            for u, v, w in self.graph:
                if dist[u] != float("Inf") and dist[u] + w < dist[v]:
                        dist[v] = dist[u] + w
 
        # Step 3: check for negative-weight cycles.  The above step 
        # guarantees shortest distances if graph doesn't contain 
        # negative weight cycle.  If we get a shorter path, then there
        # is a cycle.
 
        for u, v, w in self.graph:
                if dist[u] != float("Inf") and dist[u] + w < dist[v]:
                        print "Graph contains negative weight cycle"
                        return
                         
        # print all distance
        self.printArr(dist)
\end{lstlisting}

\newpage
\subsection{Dijkstra}
\paragraph{Référence} \cite{Algorithm3}
\begin{lstlisting}
using System;
using System.Collections.Generic;

namespace Dijkstras
{
    class Graph
    {
        Dictionary<char, Dictionary<char, int>> vertices =
            new Dictionary<char, Dictionary<char, int>>();

        public void add_vertex(char name, Dictionary<char, int> edges)
        {
            vertices[name] = edges;
        }

        public List<char> shortest_path(char start, char finish)
        {
            var previous = new Dictionary<char, char>();
            var distances = new Dictionary<char, int>();
            var nodes = new List<char>();

            List<char> path = null;

            foreach (var vertex in vertices)
            {
                if (vertex.Key == start)
                {
                    distances[vertex.Key] = 0;
                }
                else
                {
                    distances[vertex.Key] = int.MaxValue;
                }

                nodes.Add(vertex.Key);
            }

            while (nodes.Count != 0)
            {
                nodes.Sort((x, y) => distances[x] - distances[y]);

                var smallest = nodes[0];
                nodes.Remove(smallest);

                if (smallest == finish)
                {
                    path = new List<char>();
                    while (previous.ContainsKey(smallest))
                    {
                        path.Add(smallest);
                        smallest = previous[smallest];
                    }

                    break;
                }

                if (distances[smallest] == int.MaxValue)
                {
                    break;
                }

                foreach (var neighbor in vertices[smallest])
                {
                    var alt = distances[smallest] + neighbor.Value;
                    if (alt < distances[neighbor.Key])
                    {
                        distances[neighbor.Key] = alt;
                        previous[neighbor.Key] = smallest;
                    }
                }
            }

            return path;
        }
    }
}
\end{lstlisting}

\newpage
\subsection{A*}
\paragraph{Référence} \cite{Algorithm5}
\begin{lstlisting}
template<typename Location, typename Graph>
void dijkstra_search
  (Graph graph,
   Location start,
   Location goal,
   std::map<Location, Location>& came_from,
   std::map<Location, double>& cost_so_far)
{
  PriorityQueue<Location, double> frontier;
  frontier.put(start, 0);

  came_from[start] = start;
  cost_so_far[start] = 0;
  
  while (!frontier.empty()) {
    Location current = frontier.get();

    if (current == goal) {
      break;
    }

    for (Location next : graph.neighbors(current)) {
      double new_cost = cost_so_far[current] + graph.cost(current, next);
      if (cost_so_far.find(next) == cost_so_far.end()
          || new_cost < cost_so_far[next]) {
        cost_so_far[next] = new_cost;
        came_from[next] = current;
        frontier.put(next, new_cost);
      }
    }
  }
}
\end{lstlisting}