\begin{thebibliography}{99}
    \bibitem{23andMe} \emph{23andMe}, \url{https://www.23andme.com/howitworks/}

    \bibitem{AI0} Artificial Intelligence, \url{https://en.wikipedia.org/wiki/Artificial_intelligence}
    \bibitem{AI1} Virginie Mathivet, \emph{L'Intelligence Artificielle pour les développeurs Java}, eni Editions, 2015
    \bibitem{AI2} John McCarthy, \emph{Biography}, \url{http://www.computerhistory.org/fellowawards/hall/john-mccarthy/}

    \bibitem{Algorithm0} Mishadoff Thoughts, \url{http://mishadoff.com/blog/dfs-on-binary-tree-array/}
    \bibitem{Algorithm1} Koder Dojo, \emph{C\# Breadth First Search}, \url{http://www.koderdojo.com/blog/breadth-first-search-and-shortest-path-in-csharp-and-net-core}
    \bibitem{Algorithm2} Koder Dojo, \emph{C\# Depth First Search}, \url{http://www.koderdojo.com/blog/depth-first-search-algorithm-in-csharp-and-net-core}
    \bibitem{Algorithm3} mburst, \emph{Implementations of Dijkstra's shortest path algorithm in different languages}, \url{https://github.com/mburst/dijkstras-algorithm}
    \bibitem{Algorithm4} Dynamic programming, \emph{Geeks for Geeks}, \url{https://www.geeksforgeeks.org/dynamic-programming-set-23-bellman-ford-algorithm/}
    \bibitem{Algorithm5} Implementation of A*, \url{https://www.redblobgames.com/pathfinding/a-star/implementation.html}

    \bibitem{AlphaGo0} David Silver \and Aja Huang, \emph{Mastering the Game of Go with Deep Neural Networks and Tree Search}, 2016, \url{https://gogameguru.com/i/2016/03/deepmind-mastering-go.pdf}
    \bibitem{AlphaGo1} Mirek Stanek, \emph{Understanding AlphaGo}, 2017, \url{https://machinelearnings.co/understanding-alphago-948607845bb1}
    \bibitem{AlphaGo2} AlphaGo Zero: Learning from scratch, \url{https://deepmind.com/blog/alphago-zero-learning-scratch/}
    \bibitem{AlphaGo3} Seth Weidman, \emph{3 tricks that made AlphaGo Zero work}, \url{https://hackernoon.com/the-3-tricks-that-made-alphago-zero-work-f3d47b6686ef}
    \bibitem{AlphaGo4} Xavier Amatriain, \emph{What is the significance of AlphaGo Zero in AI research?}, \url{https://www.quora.com/What-is-the-significance-of-AlphaGo-Zero-in-AI-research/answer/Xavier-Amatriain}

    \bibitem{Arte0} \emph{Informations, le grand complot}, Arte, 2017, \url{https://www.arte.tv/fr/videos/074592-000-A/informations-le-grand-complot/}, \url{https://www.youtube.com/watch?v=RfybAOSlrSs}

    \bibitem{Asimov0} Isaac Asimov, \emph{Fondation}, Gnome Press, 1951.
    \bibitem{Asimov1} Isaav Asimov, \emph{Le Cycle des Robots}, Gnome Press, 1950

    \bibitem{Blockchain0} \emph{Ethereum Sharding: Introduction}, \url{https://github.com/ethereum/sharding/blob/develop/docs/doc.md}
    \bibitem{Blockchain1} \emph{Ethereum Sharding FAQ}, \url{https://github.com/ethereum/wiki/wiki/Sharding-FAQ}
    \bibitem{Blockchain2} \emph{Lightning Network Megathread}, \url{https://www.reddit.com/r/Bitcoin/comments/7pwna9/lightning_network_megathread/}
    \bibitem{Blockchain3} \emph{Republic of Estonia E-Residency}, \url{https://e-resident.gov.ee/}

    \bibitem{Brain0} Chris Chatham, \emph{10 Important Differences Between Brains and Computers}, \url{http://scienceblogs.com/developingintelligence/2007/03/27/why-the-brain-is-not-like-a-co/}
    \bibitem{Brain1} Comparaison entre cerveau humain et ordinateur \url{http://intelligence-artificielle-tpe.e-monsite.com/pages/limites-technologiques-et-ethique-de-l-ia/cerveau-humain-et-robot.html}
    
    \bibitem{Controle0} \emph{Contrôle}, Dictionnaire en ligne Larousse, \url{http://www.larousse.fr/dictionnaires/francais/contr%C3%B4le/18932}
    \bibitem{Controle1} \emph{Contrôle}, Dictionnaire historique de la langue française, Le Robert, 1992.

    \bibitem{CryptoMonnaies0} Yann Rousseau, \emph{Panique sur le marché des cryptomonnaies après de nouvelles menaces de Séoul}, \url{https://www.lesechos.fr/finance-marches/marches-financiers/0301133160429-panique-sur-le-marche-des-cryptomonnaies-apres-de-nouvelles-menaces-de-seoul-2144060.php}
    \bibitem{CryptoMonnaies1} \emph{La Chine fait plonger le cours du bitcoin}, \url{https://www.latribune.fr/economie/international/la-chine-fait-plonger-le-cours-du-bitcoin-750319.html}

    \bibitem{Damasio0} Alain Damasio, \emph{La Zone du Dehors}, Cylibris, 1999.
    \bibitem{Damasio1} Alain Damasio, \emph{Le Dehors de toutes choses}, La Volte, 2016.
    \bibitem{Damasio2} Alain Damasio, \emph{Très humain plutôt que transhumain}, 2014, \url{https://www.youtube.com/watch?v=cR0T5-a6YTc}

    \bibitem{DarkWeb0} Rayna Stamboliyska, \emph{Du sextoy au "Dark Web"}, 2017, \url{https://www.youtube.com/watch?v=zyB1eQFskf4}

    \bibitem{Deleuze0} Gilles Deleuze, \emph{Post-scriptum sur les sociétés de contrôle}, L'autre journal, 1990.

    \bibitem{Eugenisme0} \emph{Eugénisme}, CNRTL, \url{http://www.cnrtl.fr/definition/eug%C3%A9nisme}

    \bibitem{Fixe0} \emph{Fixe, Fixer}, Dictionnaire historique de la langue française, Le Robert, 1992. 

    \bibitem{Foucault0} Michel Foucault, \emph{Surveiller et punir}, Gallimard, 1975.

    \bibitem{FriesMilano0} Valentin Fries \& Vincent Milano, \emph{Manifeste de l'Éthique Informatique}, \emph{Données centralisées et intelligences interconnectées}, 2018

    \bibitem{Genetique0} Jean-Philippe Braly, \emph{CRISPR-Cas9 : le couteau suisse qui révolutionne la génétique}, \url{http://www.cite-sciences.fr/fr/ressources/science-actualites/detail/news/crispr-cas9-le-couteau-suisse-qui-revolutionne-la-genetique/}

    \bibitem{GhostInTheShell} Masamune Shirow, \emph{Ghost in the Shell}, 1989.

    \bibitem{GraphTheory0} Jessica Su, \emph{What is the difference between a tree and a forest in graph theory?}, 2018, \url{https://www.quora.com/What-is-the-difference-between-a-tree-and-a-forest-in-graph-theory}
    \bibitem{GraphTheory1} Keith Horwood, \emph{Using Graph Theory to Build a Simple Recommendation Engine in JavaScript}, \url{https://medium.com/@keithwhor/using-graph-theory-to-build-a-simple-recommendation-engine-in-javascript-ec43394b35a3}

    \bibitem{Gunnm} Yukito Kishiro, \emph{Gunnm}, 1990.

    \bibitem{Heuristics0} Jeff Bradberry, \emph{Introduction to Monte Carlo Tree Search}, \url{https://jeffbradberry.com/posts/2015/09/intro-to-monte-carlo-tree-search/}

    \bibitem{Huxley0} Aldousse Huxley, \emph{Le Meilleur des mondes}, Pocket, 1932.

    \bibitem{Internet0} Émilia Robin, Dadid Madore, Marie-Lan Nguyen et Joël Rien, \emph{Brève histoire d'Internet}, Tuteurs de l'École Normale Supérieure, 2005, \url{http://www.tuteurs.ens.fr/internet/histoire.html}
    \bibitem{Internet1} Sir Timothy John Berners-Lee, \emph{Information Management: A Proposal}, 1989, \url{https://www.w3.org/History/1989/proposal.html}
    \bibitem{Internet2} \emph{Wolfram|Alpha}, \url{https://www.wolframalpha.com/tour/}
    \bibitem{Internet3} Tim Bray, \emph{Google Memory Loss}, 2018, \url{https://www.tbray.org/ongoing/When/201x/2018/01/15/Google-is-losing-its-memory}
    \bibitem{Internet4} \emph{Tor Project}, \url{https://www.torproject.org/}
    \bibitem{Internet5} \emph{Tor: Onion Service Protocol}, \url{https://www.torproject.org/docs/onion-services}
    \bibitem{Internet6} \emph{Dutch National Prosecution Service and Police launch hidden service in global darknet enforcement operation}, DeepDotWeb, 2016, \url{https://www.deepdotweb.com/2016/10/31/dutch-national-prosecution-service-police-launch-hidden-service-global-darknet-enforcement-operation/}
    \bibitem{Internet7} Runa Sandvik, \emph{The New York Times is Now Available as a Tor Onion Service}, The New York Times, 2017, \url{https://open.nytimes.com/https-open-nytimes-com-the-new-york-times-as-a-tor-onion-service-e0d0b67b7482}
    
    \bibitem{Instinct0} \emph{Instinct}, Dictionnaire historique de la langue française, Le Robert, 1992. 

    \bibitem{Kirin970} Cassim Ketfi, \emph{Kirin970 : le processeur du Huawei Mate 10 présenté à l'IFA 2017}, \url{http://www.frandroid.com/marques/huawei/456511_processeur-soc-10-nm-caracteristiques-huawei-kirin-970-mate-10-ifa-2017}

    \bibitem{Klein0} Naomi Klein, \emph{No Logo}, Actes Sud, 1999.

    \bibitem{Kubrick0} Stanley Kubrick, \emph{2001, l'Odyssée de l'espace}, 1968

    \bibitem{Language0} ELIZA, \url{https://en.wikipedia.org/wiki/ELIZA}

    \bibitem{MachineLearning0} Jeroen Moons, \emph{A casual intro to Machine Learning}, Intracto, 2017, \url{https://blog.intracto.com/a-casual-intro-to-machine-learning}
    \bibitem{MachineLearning1} Kendrick Tan, \emph{Indexing faces on Instagram}, Kendrick.co, 2017, \url{https://kndrck.co/indexing-faces-on-instagram.html}
    \bibitem{MachineLearning2} Vishal Maini, \emph{Machine Learning for Humans}, \url{https://medium.com/machine-learning-for-humans/supervised-learning-740383a2feab}
    \bibitem{MachineLearning3} Sunil Ray, \emph{7 types of Regression Techniques}, \url{https://www.analyticsvidhya.com/blog/2015/08/comprehensive-guide-regression/}
    \bibitem{MachineLearning4} R, \emph{Linear Regression}, \url{http://r-statistics.co/Linear-Regression.html}
    \bibitem{MachineLearning5} John Glover, \emph{An introduction to Generative Adversarial Networks}, \url{http://blog.aylien.com/introduction-generative-adversarial-networks-code-tensorflow/}
    \bibitem{MachineLearning6} Ali Reza Kohani, \emph{Regression vs Classification}, \url{https://medium.com/@ali_88273/regression-vs-classification-87c224350d69}

    \bibitem{Marx0} Karl Marx, \emph{Manuscrits de 1844}, 1932, \url{http://www.jcmullen.fr/1844.MP3}
    \bibitem{Marx1} Karl Marx \and Friedrich Engels, \emph{Manifeste du Parti communiste}, 1848.

    \bibitem{Metaheuristics0} S. Le Digabel, \emph{Ecole Polytechnique de Montréal}, \emph{Introduction aux metaheuristiques}, \url{https://www.gerad.ca/Sebastien.Le.Digabel/MTH6311/5_Introduction_Metaheuristiques.pdf}

    \bibitem{Microservices0} Martin Fowler \& James Lewis, \emph{Microservices: a definition of this new architectural term}, 2014, \url{https://martinfowler.com/articles/microservices.html}
    \bibitem{Microservices1} Dave Kerr, \emph{The Death of Microservice Madness in 2018}, 2018, \url{http://www.dwmkerr.com/the-death-of-microservice-madness-in-2018/}
    \bibitem{Microservices2} Vineet Badola, \emph{Microservices architecture: advantages and drawbacks}, Cloud Academy, 2015, \url{https://cloudacademy.com/blog/microservices-architecture-challenge-advantage-drawback/}
    \bibitem{Microservices3} \emph{The Great Microservices vs Monolithic Apps Twitter melee}, High Scalability, 2014, \url{http://highscalability.com/blog/2014/7/28/the-great-microservices-vs-monolithic-apps-twitter-melee.html}
    \bibitem{Microservices4} Robert Annett, \emph{What is a Monolith?}, Coding the Architecture, 2014, \url{http://www.codingthearchitecture.com/2014/11/19/what_is_a_monolith.html}
    \bibitem{Microservices5} \emph{Microservices vs Monolithic Architecture}, MuleSoft, \url{https://www.mulesoft.com/resources/api/microservices-vs-monolithic}
    \bibitem{Microservices6} Chris Richardson, \emph{Pattern: Monolithic Architecture}, Microservices.io, 2017, \url{http://microservices.io/patterns/monolithic.html}
    \bibitem{Microservices7} Nick Craver, \emph{Stack Overflow: the Architecture - 2016 Edition}, \url{https://nickcraver.com/blog/2016/02/17/stack-overflow-the-architecture-2016-edition/}

    \bibitem{Moore0} Thomas Moore, \emph{Utopia}, 1516.

    \bibitem{MultiLearning0} Ronan Collobert \and Jason Weston, \emph{A unified architecture for natural language processing: deep neural networks with multitask learning}, \url{https://dl.acm.org/citation.cfm?doid=1390156.1390177}
    \bibitem{MultiLearning1} Li Deng, Geoffrey Hinton \and Brian Kingsbury, \emph{New types of deep neural network learning for speech recognition and related applications: an overview}, \url{http://ieeexplore.ieee.org/document/6639344/}
    \bibitem{MultiLearning2} Bharath Ramsundar, Steven Kearnes, Patrick Riley, Dale Webster, David Konerding \and Vijay Pande, \emph{Massively Multitask Networks for Drug Discovery}, \url{https://arxiv.org/abs/1502.02072}

    \bibitem{Negri0} Antonio Negri \and Michael Hardt, \emph{Empire}, Exils, 2000.

    \bibitem{NetNeutrality0} Le Monde, \emph{Les Etats-Unis abrogent la neutralité du Net, un principe fondateur d’Internet}, 15 Décembre 2017, \url{http://www.lemonde.fr/pixels/article/2017/12/14/les-etats-unis-abrogent-la-neutralite-du-net-un-principe-fondateur-d-internet_5229906_4408996.html}
    \bibitem{NetNeutrality1} Camille Gévaudan, \emph{La neutralité du Net pour les nuls}, \url{http://www.liberation.fr/ecrans/2014/09/12/la-neutralite-du-net-pour-les-nuls_1098902}

    \bibitem{NeuralNets0} Sebastian Ruder, \emph{An Overview of Multi-Task Learning in Deep Neural Networks}, \url{http://ruder.io/multi-task/index.html#introduction}
    \bibitem{NeuralNets1} Caruana, \emph{Multitask Learning}, \url{https://dl.acm.org/citation.cfm?id=262872}

    \bibitem{Nietzsche0} Friedrich Nietzsche, \emph{Ainsi parlait Zarathoustra}, 1885.

    \bibitem{Optimums0} \emph{Maxima and Minima of functions} \url{https://www.mathsisfun.com/algebra/functions-maxima-minima.html}

    \bibitem{Orwell0} George Orwell, \emph{1984}, Secker and Warburg, 1949.

    \bibitem{Panoptique0} \emph{Panoptique}, Dictionnaire en ligne Larousse, \url{http://www.larousse.fr/dictionnaires/francais/panoptique/57663}
    \bibitem{Panoptique1} Anne Chemin, \emph{Prisons : du panoptique de Bentham à Michel Foucault}, Le Monde, 05/06/2014, \url{http://www.lemonde.fr/culture/article/2014/06/05/prisons-du-panoptique-de-bentham-a-michel-foucault_4432900_3246.html}
    \bibitem{Panoptique2} Jérémy Bentham, \emph{Panopticon}, 1791.

    \bibitem{Partage0} \emph{Partage, Partager}, Dictionnaire historique de la langue française, Le Robert, 1992.

    \bibitem{Pathway0} \emph{Pathway Genomics}, \url{https://www.pathway.com/products/}

    \bibitem{PersonalData0} \emph{Donnée personnelle}, CNIL, \url{https://www.cnil.fr/fr/definition/donnee-personnelle}

    \bibitem{Pervers0} \emph{Pervertir, Pervers}, Dictionnaire historique de la langue française, Le Robert, 1992.

    \bibitem{Pieces0} \emph{RFID: la police totale}, Pièces et main d'oeuvre, 2006, \url{http://www.piecesetmaindoeuvre.com/IMG/pdf/RFID_la_police_totale.pdf}

    \bibitem{ProgrammableCity0} \emph{The Programmable City}, Maynooth University, 2017, \url{http://progcity.maynoothuniversity.ie/}
    \bibitem{ProgrammableCity1} \emph{Les données permettent-elles de piloter la ville intelligente ?}, Le Monde, 2017, \url{http://internetactu.blog.lemonde.fr/2017/06/18/les-donnees-permettent-elles-de-piloter-la-ville/}
    \bibitem{ProgrammableCity2} \emph{Big Data and Machine Learning harnessed for crime prevention}, Atelier BNP Paribas, 2015, \url{https://atelier.bnpparibas/en/smart-city/article/big-data-machine-learning-harnessed-crime-prevention}
    \bibitem{ProgrammableCity3} \emph{SideWalk Labs}, \url{https://www.sidewalklabs.com/}

    \bibitem{PsychoPass} Katsuyuki Motohiro, \emph{Psycho-Pass}, 2012.

    \bibitem{Rabhi0} Pierre Rabhi, \emph{Vers la sobriété heureuse}, Babel, 2013.
    \bibitem{Rabhi1} Pierre Rabhi, \emph{Y a-t-il une vie avant la mort ?}, 2011, \url{https://www.youtube.com/watch?v=HyNinbbzGuE}

    \bibitem{Richesses0} \emph{La concentration des richesses dans le monde en graphiques}, Le Monde, 2015, \url{http://www.lemonde.fr/les-decodeurs/article/2015/01/19/la-concentration-des-richesses-dans-le-monde-en-graphiques_4558914_4355770.html}

    \bibitem{Rufin0} Jean-Christophe Rufin, \emph{Globalia}, Gallimard, 2003.

    \bibitem{Rust0} \emph{The Rust Programming Language}, \url{https://www.rust-lang.org}

    \bibitem{Sartre0} Jean-Paul Sartre, \emph{Huis clos}, Gallimard, 1944
    
    \bibitem{Scalability0} A. Keromytis \& J. Smith, \emph{Requirements for Scalable Access Control and Security Management Architecture}, Columbia University \& University of Pennsylvania, 2007, \url{https://www.cs.columbia.edu/~angelos/Papers/2007/toit-access.pdf}

    \bibitem{Smartphone0} Benjamin Hue, \emph{Comment l'intelligence artificielle transforme les téléphones en superphones}, \url{http://www.rtl.fr/actu/futur/comment-l-intelligence-artificielle-transforme-les-telephones-en-superphones-7790496261}
    \bibitem{Smartphone1} Paolo Garoscio, \emph{Honor View 10 : un smartphone dopé à l'IA}, \url{http://www.clubic.com/smartphone/android/actualite-839982-honor-view-10-smartphone-dope-ia.html}

    \bibitem{SocialMedia0} Jesse Singal, \emph{Social media is making us dumber. Here's exhibit A.}, The New York Times, 2018, \url{https://www.nytimes.com/2018/01/11/opinion/social-media-dumber-steven-pinker.html}
    \bibitem{SocialMedia1} \emph{Social sharing - Statistics \and Facts}, Statista, 2016, \url{https://www.statista.com/topics/2539/social-sharing/}
    \bibitem{SocialMedia2} \emph{De la liberté à la désinformation : la nouvelle image des réseaux sociaux}, Capital, 2017, \url{https://www.capital.fr/lifestyle/symbole-de-liberte-ou-vecteur-de-desinformation-limage-des-reseaux-sociaux-a-change-1247192}
    \bibitem{SocialMedia3} Paul Roberts, \emph{11 really useful social media statistics for 2018}, Our Social Times, \url{https://oursocialtimes.com/11-really-useful-social-media-statistics-for-2018/}
    \bibitem{SocialMedia4} L. Mitrou, M. Kandias, V. Stavrou \& D. Gritzalis, \emph{Social media profiling : a Panopticon or Omniopticon tool?}, 2014, \url{https://www.infosec.aueb.gr/Publications/2014-SSN-Privacy%20Social%20Media.pdf}

    \bibitem{Surveillance0} \emph{Veiller, Surveiller, Surveillance}, Dictionnaire historique de la langue française, Le Robert, 1992.

    \bibitem{TechnoSocio0} Dominique Vinck, \emph{Humanités numériques, la culture face aux nouvelles technologies}, 2016.
    \bibitem{TechnoSocio1} Carsten Wilhelm, \emph{Vive la technologie ? : Traité de bricolage réfléchi pour épris de liberté Ed. 1}, 2015.

    \bibitem{Therapy0} \emph{Eliza, computer therapist} \url{http://www.manifestation.com/neurotoys/eliza.php3}

    \bibitem{Turing0} \emph{The Turing Test}, \url{http://www.psych.utoronto.ca/users/reingold/courses/ai/turing.html}

    \bibitem{Universel0} \emph{Univers, Universel}, Dictionnaire historique de la langue française, Le Robert, 1992.

    \bibitem{University0} Magalie Fromont, \emph{Apprentissage Statistique}, \url{https://perso.univ-rennes2.fr/system/files/users/fromont_m/Apprentissage_1516_Lasso.pdf}
    \bibitem{University1} PennState Eberly College of Science, \emph{Stepwise Regression}, \url{https://onlinecourses.science.psu.edu/stat501/node/329}
\end{thebibliography}