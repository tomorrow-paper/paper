\begin{thebibliography}{99}
    \bibitem{AlphaGo0} David Silver \and Aja Huang, \emph{Mastering the Game of Go with Deep Neural Networks and Tree Search}, 2016, \url{https://gogameguru.com/i/2016/03/deepmind-mastering-go.pdf}
    \bibitem{AlphaGo1} Mirek Stanek, \emph{Understanding AlphaGo}, 2017, \url{https://machinelearnings.co/understanding-alphago-948607845bb1}

    \bibitem{Arte0} \emph{Informations, le grand complot}, Arte, 2017, \url{https://www.arte.tv/fr/videos/074592-000-A/informations-le-grand-complot/}

    \bibitem{Asimov0} Isaac Asimov, \emph{Fondation}, Gnome Press, 1951.    
    
    \bibitem{Controle0} \emph{Contrôle}, Dictionnaire en ligne Larousse, \url{http://www.larousse.fr/dictionnaires/francais/contr%C3%B4le/18932}
    \bibitem{Controle1} \emph{Contrôler}, Wiktionnaire, \url{https://fr.wiktionary.org/wiki/contr%C3%B4ler}

    \bibitem{Damasio0} Alain Damasio, \emph{La Zone du Dehors}, Cylibris, 1999.
    \bibitem{Damasio1} Alain Damasio, \emph{Le Dehors de toutes choses}, La Volte, 2016.
    \bibitem{Damasio2} Alain Damasio, \emph{Très humain plutôt que transhumain}, 2014, \url{https://www.youtube.com/watch?v=cR0T5-a6YTc}

    \bibitem{DarkWeb0} Rayna Stamboliyska, \emph{Du sextoy au "Dark Web"}, 2017, \url{https://www.youtube.com/watch?v=zyB1eQFskf4}

    \bibitem{Deleuze0} Gilles Deleuze, \emph{Post-scriptum sur les sociétés de contrôle}, L'autre journal, 1990.

    \bibitem{Foucault0} Michel Foucault, \emph{Surveiller et punir}, Gallimard, 1975.

    \bibitem{GhostInTheShell} Masamune Shirow, \emph{Ghost in the Shell}, 1989.

    \bibitem{Gunnm} Yukito Kishiro, \emph{Gunnm}, 1990.

    \bibitem{Huxley0} Aldousse Huxley, \emph{Le Meilleur des mondes}, Pocket, 1932.

    \bibitem{Klein0} Naomi Klein, \emph{No Logo}, Actes Sud, 1999.

    \bibitem{MachineLearning0} Jeroen Moons, \emph{A casual intro to Machine Learning}, Intracto, 2017, \url{https://blog.intracto.com/a-casual-intro-to-machine-learning}
    \bibitem{MachineLearning1} Kendrick Tan, \emph{Indexing faces on Instagram}, Kendrick.co, 2017, \url{https://kndrck.co/indexing-faces-on-instagram.html}

    \bibitem{Marx0} Karl Marx, \emph{Manuscrits de 1844}, 1932, \url{http://www.jcmullen.fr/1844.MP3}
    \bibitem{Marx1} Karl Marx \and Friedrich Engels, \emph{Manifeste du Parti communiste}, 1848.

    \bibitem{Moore0} Thomas Moore, \emph{Utopia}, 1516.

    \bibitem{Negri0} Antonio Negri \and Michael Hardt, \emph{Empire}, Exils, 2000.

    \bibitem{Nietzsche0} Friedrich Nietzsche, \emph{Ainsi parlait Zarathoustra}, 1885.

    \bibitem{Orwell0} George Orwell, \emph{1984}, Secker and Warburg, 1949.

    \bibitem{PersonalData0} \emph{Donnée personnelle}, CNIL, \url{https://www.cnil.fr/fr/definition/donnee-personnelle}

    \bibitem{Pieces0} \emph{RFID: la police totale}, Pièces et main d'oeuvre, 2006, \url{http://www.piecesetmaindoeuvre.com/IMG/pdf/RFID_la_police_totale.pdf}

    \bibitem{ProgrammableCity0} \emph{The Programmable City}, Maynooth University, 2017, \url{http://progcity.maynoothuniversity.ie/}
    \bibitem{ProgrammableCity1} \emph{Les données permettent-elles de piloter la ville intelligente ?}, Le Monde, 2017, \url{http://internetactu.blog.lemonde.fr/2017/06/18/les-donnees-permettent-elles-de-piloter-la-ville/}
    \bibitem{ProgrammableCity2} \emph{Big Data and Machine Learning harnessed for crime prevention}, Atelier BNP Paribas, 2015, \url{https://atelier.bnpparibas/en/smart-city/article/big-data-machine-learning-harnessed-crime-prevention}
    \bibitem{ProgrammableCity3} \emph{SideWalk Labs}, \url{https://www.sidewalklabs.com/}

    \bibitem{PsychoPass} Katsuyuki Motohiro, \emph{Psycho-Pass}, 2012.

    \bibitem{Rabhi0} Pierre Rabhi, \emph{Vers la sobriété heureuse}, Babel, 2013.
    \bibitem{Rabhi1} Pierre Rabhi, \emph{Y a-t-il une vie avant la mort ?}, 2011, \url{https://www.youtube.com/watch?v=HyNinbbzGuE}

    \bibitem{Rufin0} Jean-Christophe Rufin, \emph{Globalia}, Gallimard, 2003.

    \bibitem{TechnoSocio0} Dominique Vinck, \emph{Humanités numériques, la culture face aux nouvelles technologies}, 2016.
    \bibitem{TechnoSocio1} Carsten Wilhelm, \emph{Vive la technologie ? : Traité de bricolage réfléchi pour épris de liberté Ed. 1}, 2015.

    \bibitem{Panoptique0} \emph{Panoptique}, Dictionnaire en ligne Larousse, \url{http://www.larousse.fr/dictionnaires/francais/panoptique/57663}
    \bibitem{Panoptique1} Anne Chemin, \emph{Prisons : du panoptique de Bentham à Michel Foucault}, Le Monde, 05/06/2014, \url{http://www.lemonde.fr/culture/article/2014/06/05/prisons-du-panoptique-de-bentham-a-michel-foucault_4432900_3246.html}
    \bibitem{Panoptique2} Jérémy Bentham, \emph{Panopticon}, 1791.

    \bibitem{Internet0} Émilia Robin, Dadid Madore, Marie-Lan Nguyen et Joël Rien, \emph{Brève histoire d'Internet}, Tuteurs de l'École Normale Supérieure, 2005, \url{http://www.tuteurs.ens.fr/internet/histoire.html}
    \bibitem{Internet1} Sir Timothy John Berners-Lee, \emph{Information Management: A Proposal}, 1989, \url{https://www.w3.org/History/1989/proposal.html}
    \bibitem{Internet2} Artificial Intelligence, \url{https://en.wikipedia.org/wiki/Artificial_intelligence}
    \bibitem{Internet3} Door Jeroen Moons, \emph{A casual intro to Machine Leaning}, \url{https://blog.intracto.com/a-casual-intro-to-machine-learning}
    \bibitem{Internet4} Chris Chatham, \emph{10 Important Differences Between Brains and Computers}, \url{http://scienceblogs.com/developingintelligence/2007/03/27/why-the-brain-is-not-like-a-co/}
    \bibitem{Internet5} Comparaison entre cerveau humain et ordinateur \url{http://intelligence-artificielle-tpe.e-monsite.com/pages/limites-technologiques-et-ethique-de-l-ia/cerveau-humain-et-robot.html}
\end{thebibliography}